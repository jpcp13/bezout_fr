\documentclass{standalone}
% Preamble
\begin{document}


\section{Cas multivariable}
\label{multivariable}

Dans le cas univariable, examiné à la section précédente, la structure de $A$ se compose d'une part d'une base, en l'occurence la base des monômes, d'autre part de la matrice compagnon, exprimant l'endomorphisme de multiplication par $x$ dans cette base. Ces deux éléments peuvent être obtenus soit directement par lecture des coefficients de $f$, soit à partir des matrices de Bezout $B(1), B(x)$.

Dans le le cas multivariable, développé dans cette section, ni une base ni les matrices compagnon (matrices des opérateurs
$x_j : \left\vert
\begin{array}{c}
A \mapsto A \\
h \mapsto x_jh
\end{array}
\right.$ dans la base) ne sont visibles directement sur les coefficients des polynômes de départ. En revanche nous allons montrer comment construire des matrices de Bezout $B(1), B(x_1), \cdots, B(x_n)$ à partir desquelles on peut obtenir une base et les matrices compagnon $X_j$ associées à la base obtenue. Commençons par fixer le cadre de travail.
Pour $n$ polynômes $f_1,\cdots, f_n$ en $n$ variables $x_1,\cdots, x_n$ à coefficients dans $\C$, considérons :
\begin{itemize}
\item $\C[x]$ l'anneau des polynômes en les variables $x = x_1,\cdots, x_n$
\item $\langle f \rangle$ l'idéal généré par la famille  $f = f_1,\cdots, f_n$
\item $V(f) = \{x \in \C^n : f(x) = 0\}$ la variété associée à $\langle f\rangle$
\item $A = \C[x]/\langle f\rangle$ l'algèbre quotient
\end{itemize}
Nous supposerons dorénavant que l'idéal $\langle f\rangle$ est {\bf zéro-dimensionel}, c'est-à-dire que $V(I)$ est fini ou, de façon équivalente \cite[p.~234]{clo}, que $A$ est de {\bf dimension finie} en tant qu'espace vectoriel sur $\C$. Ceci est bien sûr toujours le cas lorsque $n = 1$.

\subsection{Construction des polynômes et des matrices de Bezout}

\subsubsection{Extension de la Définition \ref{def_bez} au cas multivariable}


\begin{defn}
Soit $x^\gamma = x_1^{\gamma_1}\cdots x_n^{\gamma_n} \in \C[x]$ un monôme.
Introduisons un nouveau jeu de variables $y = y_1,\cdots, y_n$ et considérons, pour tous $i, j = 1\cdots n$, le rapport
\begin{equation}
\label{finite_diff}
\delta_{i,j}(x^\gamma) = \dfrac{y_j^{\gamma_j}f_i(y_1,\cdots, y_{j-1},x_j,\cdots,x_n) - x_j^{\gamma_j}f_i(y_1,\cdots,y_j,x_{j+1},\cdots,x_n)}{x_j - y_j}
\end{equation}
qui est un polynôme en les variables $x, y$. Nous obtenons une matrice de différences finies $\Delta(x^\gamma) = (\delta_{ij}(x^\gamma))_{ij}$, qui est à la matrice jacobienne ce que le taux d'accroissement est à la dérivée.
Le {\bf polynôme de Bezout} du monôme $x^\gamma$ est par définition
\begin{equation}
	\delta(x^\gamma) = det(\Delta(x^\gamma))
\end{equation}
qui est un élément de $\C[x, y]$. Pour un polynôme général $g = \sum_\gamma g_\gamma x^\gamma \in \C[x]$, on étend la définition précédente par linéarité $\delta(g) = \sum_\gamma g_\gamma \delta(x^\gamma)$.
En développant $\delta(g) = \sum_{\alpha,\beta} b_{\alpha\beta} x^\alpha y^\beta$ comme une somme de monômes de $\C[x, y]$, et en notant $\bold{x}$ et $\bold{y}$ les familles de tous les monômes en $x$ et $y$ apparaissant dans ce développement, nous définissons la {\bf matrice de Bezout} $B(g) = [b_{\alpha\beta}]$, c'est à dire que l'on a la relation suivante, similaire à la relation~(\ref{pmB}), entre polynôme et matrice de Bezout
\begin{equation}
	\delta(g) = \bold{x} B(g) \bold{y}^T
\end{equation}
\end{defn}

\begin{exmp}
\label{bez_multi}
Fixons $n = 2$ et considérons $f_1 = x_1^2 + x_1x_2^2 - 1, f_2 = x_1^2x_2 + x_1$.
Nous allons calculer les matrices de Bezout $B(1), B(x_1), B(x_2)$,  qui vont servir à la construction des matrices compagnon $X_1, X_2$, comme nous le verrons plus loin. Pour commencer, calculons à partir des formules (\ref{finite_diff})
\begin{align}
\Delta(1) &=
\begin{pmatrix}
x_1 + x_2^2 + y_1 & x_2y_1 + y_1y_2 \\
1 + x_1x_2 + x_2y_1 & y_1^2
\end{pmatrix} \nonumber  \\
\Delta(x_1) &=
\begin{pmatrix}
1 + x_1y_1 & x_2y_1 + y_1y_2 \\
1 + x_1x_2 + x_2y_1 & y_1^2
\end{pmatrix} \nonumber  \\
\Delta(x_2) &=
\begin{pmatrix}
x_1 + x_2^2 + y_1 & 1 - y_1^2 + x_2y_1y_2 \\
1 + x_1x_2 + x_2y_1  & -y_1
\end{pmatrix} \nonumber
\end{align}
dont le déterminant fournit les polynômes de Bezout
\begin{align}
\delta(1) &= -x_2y_1 - x_1x_2^2y_1 + x_1y_1^2 + y_1^3 - y_1y_2 - x_1x_2y_1y_2 - x_2y_1^2y_2 \nonumber \\
\delta(x_1) &=  y_1^2 - x_1x_2^2y_1^2 + x_1y_1^3 - x_1x_2y_1^2y_2 \nonumber \\
\delta(x_2) &= -1 - x_1x_2 - x_1y_1 -x_2y_1 - x_2^2y_1 + x_1x_2y_1^2 + x_2y_1^3 - x_2y_1y_2 - x_1x_2^2y_1y_2 - x_2^2y_1^2y_2\nonumber
\end{align}
Les familles de mônomes apparaissant dans ces polynômes sont
$\bold{x} = (1, x_2, x_2^2, x_1, x_1x_2, x_1x_2^2)$ et $\bold{y} = (1, y_1, y_1y_2, y_1^2, y_1^2y_2, y_1^3)$.
Les polynômes de Bezout s'écrivent sous forme de tableaux faisant apparaitre les matrices de Bezout
$$\begin{array}{c|cccccc}
	\delta(1) & 1 & y_1 & y_1y_2 & y_1^2 & y_1^2y_2 & y_1^3 \\
	\hline
	1 &  &  & -1 &  &  & 1\\
	x_2 &  & -1 &  &  & -1 & \\
	x_2^2 &  &  &  &  &  & \\
	x_1 &  &  &  & 1 &  & \\
	x_1x_2 &  &  & -1 &  &  & \\
	x_1x_2^2 &  & -1 &  &  &  &
\end{array}$$

$$\begin{array}{c|cccccc}
	\delta(x_1) & 1 & y_1 & y_1y_2 & y_1^2 & y_1^2y_2 & y_1^3 \\
	\hline
	1 &  &  &  & 1 &  & \\
	x_2 &  &  &  &  &  & \\
	x_2^2 &  &  &  &  &  & \\
	x_1 &  &  &  &  &  & 1\\
	x_1x_2 &  &  &  &  & -1 & \\
	x_1x_2^2 &  &  &  & -1 &  &
\end{array}
\hspace{0.2cm}
\begin{array}{c|cccccc}
	\delta(x_2) & 1 & y_1 & y_1y_2 & y_1^2 & y_1^2y_2 & y_1^3 \\
	\hline
	1 & -1 &  &  &  &  & \\
	x_2 &  & -1 & -1 &  &  & 1\\
	x_2^2 &  & -1 &  &  & -1 & \\
	x_1 &  & -1 &  &  &  & \\
	x_1x_2 & -1 &  &  & 1 &  & \\
	x_1x_2^2 &  &  & -1 &  &  &
\end{array}$$

\end{exmp}
\begin{rem}
Ici, contrairement au cas univariable, les listes $\bold{x}$ et $\bold{y}$ ne sont pas des bases de $A$. Nous verrons plus loin qu'elles sont cependant génératrices et comment on peut en extraire des bases.
\end{rem}

\subsubsection{Calcul effectif des matrices de Bezout}
Dans l'exemple précédent, les polynômes de Bezout s'obtiennent en calculant le déterminant des matrices $\Delta(1), \Delta(x_1), \Delta(x_2)$, qui sont de taille $2$ et dont les coefficients sont des polynômes en $x_1, x_2$. Si le nombre de variables $n$ ou le degré des polynômes $f_i$ augmente alors ce calcul peut devenir difficile car les coefficients des matrices $\Delta(x_k)$ ne sont pas numériques et on ne peut donc pas appliquer la méthode du pivot de Gauss. Un moyen de contourner cette difficulté est de procéder par évaluation-interpolation :
\begin{enumerate}
\item
on estime à priori l'ensemble des monômes qui vont apparaitre dans le résultat $\delta(x^\gamma)$
\item
on évalue $\Delta(x_k)$ sur un ensemble adéquat $U \times V$ de multi-points de Fourier $u = (u_1,\cdots, u_n) \in U$ et $v = (v_1,\cdots, v_n) \in V$
\item
pour chaque point $(u, v) \in U\times V$ on calcule le déterminant numérique de $\Delta(x_k)(u, v)$ par la méthode du pivot de Gauss.
\item
pour finir on interpole l'ensemble des valeurs obtenues par le polynôme cherché $\delta(x_k)$.
\end{enumerate}
Pour implémenter cet algorithme concrètement on doit préciser l'ensemble des monômes de $\delta(x_k)$ ainsi que les points de Fourier utilisés pour l'évaluation de $\delta(x_k)$. Prenons l'exemple d'un système polynomial $f$ de multidegré $(d_1, \cdots, d_n)$, c'est-à-dire que pour tous $i, j = 1..n$ le degré de $f_i$ en la variable $x_j$ est inférieur ou égal à $d_j$. Fixons un entier $k$ compris entre $0$ et $n$ et adoptons la convention que $x_0 = 1$. On voit facilement que $\delta(x_k)$, polynôme en $x, y$, est de multidegré $(d_1, 2d_2, \cdots, nd_n)$ en $x$ et de multidegré $(nd_1, (n-1)d_2, \cdots, d_n)$ en $y$.
 Pour l'évaluation de $\delta(x_k)$ aux points de Fourier $(u, v) \in U\times V$ nous choisirons donc $U = \prod_{j=1..n} U_j$ o\`u $U_j$ est l'ensemble des racines complexes de $X^{jd_j} - 1$. De même nous choisirons $V = \prod_{j=1..n} V_j$ de façon que $U_j$ et $V_j$ soient disjoints, afin que le dénominateur ne s'annule jamais dans la formule (\ref{finite_diff}). Ceci est réalisé par exemple lorsque $V_j$ est l'ensemble des racines complexes de $X^{(n-j+1)d_j} - \theta_j$ avec $\theta_j = e^{i\pi/j}$. Les considérations précédentes nous permettent maintenant d'écrire l'algorithme \ref{fourierPoints}fournissant les ensembles $U$ et $V$.
\begin{algorithm}
\caption{Construction des ensembles de points de Fourier servant à l'évaluation du polynôme de Bezout $\delta(x_k)$}\label{fourierPoints}
\begin{algorithmic}
\Function{fourierPoints}{$d$} \Comment{multidegré $d = (d_1,\cdots,d_n)$}
\For{$j=1..n$} \Comment{construction des facteurs $U_j, V_j$}
\State $U_j \gets$ ensemble des racines de $X^{jd_j}-1$
\State $V_j \gets$ ensemble des racines de $X^{(n-j+1)d_j}-e^{i\pi/j}$
\EndFor
\State $U \gets \prod_{j=1..n}U_j$
\State $V \gets \prod_{j=1..n}V_j$
\State \textbf{return} $U, V$
\EndFunction
\end{algorithmic}
\end{algorithm}
Les ensembles de points de Fourier $U$ et $V$ peuvent alors être utilisés par l'algorithme \ref{evaluationMatrix} pour construire la matrice d'évaluation du polynôme de Bezout.
\begin{algorithm}[H]
\caption{Construction de la matrice d'évaluation du polynôme de Bezout $\delta(x_k)$}
\label{evaluationMatrix}
\begin{algorithmic}
\Function{evaluation}{$f, k$} \Comment{$f = (f_1,\cdots,f_n)$ système polynomial}
\State $U, V \gets \textsc{fourierPoints}(d)$ \Comment{$d$ multidegré de $f$}
\State $D \gets \prod_{j=1..n}jd_j$
\State $C^{(k)} \gets \textsc{zeros}(D, D)$
\For{$(u, v) \in U\times V$}
      \State $\Delta \gets \textsc{zeros}(n, n)$
   		\For{$i, j=1..n$}
      		\State $\Delta_{i,j} \gets \delta_{i,j}(x_k)(u, v)$ \Comment{$\delta_{i,j}(x_k)$ défini à la formule (\ref{finite_diff})}
   		\EndFor
		\State $C^{(k)}_{u, v} \gets \textsc{det}(\Delta)$
	\EndFor
\State \textbf{return} $C^{(k)}$
\EndFunction
\end{algorithmic}
\end{algorithm}
Rappelons que la matrice de Bezout $B^{(k)} = \left[b^{(k)}_{\alpha\beta}\right]_{\alpha\beta}$ est définie par $\delta^{(k)}(x, y) = \sum_{\alpha,\beta} b^{(k)}_{\alpha\beta} x^\alpha y^\beta$. On a donc $C^{(k)}_{u,v} = \delta^{(k)}(u, v) = \sum_{\alpha,\beta} b^{(k)}_{\alpha\beta} u^\alpha v^\beta$, ce qui s'écrit comme produit de matrices
$\left[C^{(k)}_{u,v}\right]_{u,v} = \left[u^\alpha\right]_{u,\alpha} \left[b^{(k)}_{\alpha,\beta}\right]_{\alpha, \beta} \left[v^\beta\right]_{v, \beta}^T$. Définissons alors les matrices de Fourier $F_u = \left[ u^\alpha \right]_{u, \alpha}$
 et $F_v = \left[ v^\beta \right]_{v, \beta}$. On obtient la relation d'évaluation-interpolation entre les matrices $B^{(k)}$ et $C^{(k)}$
$$C^{(k)} = F_uB^{(k)} F_v^T$$
 Grâce au choix des points de Fourier fait dans l'algorithme \ref{fourierPoints} les matrices $F_u$ et $F_v$ sont unitaires et la matrice $B^{(k)}$ s'obtient alors facilement par la relation
 \begin{equation}
 B^{(k)} = F_u^*C^{(k)} \overline{F_v}
 \end{equation}
La construction des matrices de Bezout, inspirée des considérations précédentes, a été implémentée en Numpy et publiée sur le site \cite{jp_code}.

\end{document}
