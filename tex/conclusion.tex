\documentclass{standalone}
% Preamble
\begin{document}
\section{Conclusion et perspectives}
Nous avons proposé une méthode de résolution numérique des systèmes polynômiaux en intersection complète. Cette méthode utilise exclusivement des techniques d'algèbre linéaire numérique. Le principe de la méthode repose sur une conjecture de nature algébrique mais il est facile de tester si les racines obtenues sont numériquement correctes et les racines non satisfaisantes peuvent être écartées.


\end{document}
