\documentclass{article}
\usepackage{standalone}
\usepackage[utf8]{inputenc}
\usepackage{amsmath, scalefnt}
\usepackage{amsthm}
\usepackage{amssymb}
\usepackage{amsfonts}
\usepackage{graphicx}
\usepackage[francais]{babel}
\usepackage{hyperref}
% \input def_algolight.tex
\usepackage[ruled]{algorithm}
\usepackage{algorithmicx}
\usepackage{algpseudocode}

\usepackage[margin=2cm]{geometry}
\usepackage{adjustbox}
\usepackage{multirow}

%\usepackage{wrapfig}
%\usepackage{geometry}
%\geometry{ hmargin=2.8cm, vmargin=3cm }
%\usepackage{amsmath}
%\usepackage{mathdots}
\theoremstyle{plain}% default
\newtheorem{thm}{Théorème}
\newtheorem{conj}{Conjecture}
\newtheorem{lem}{Lemme}
\newtheorem{prop}{Proposition}
\newtheorem{cor}{Corollaire}
\theoremstyle{definition}
\newtheorem{defn}{Définition}
\newtheorem{exmp}{Example}
\theoremstyle{remark}
\newtheorem*{rem}{Remarque}
\newtheorem*{note}{Note}
\newtheorem*{case}{Case}

\def\R{\mathbb{R}}
\newcommand{\vecb}[1]{\pmb{#1}}
\newcommand{\VECAXPY}{\textsc{VecAxpy}}
\newcommand{\cA}{\mathcal{A}}
\newcommand{\cM}{\mathcal{M}}
\newcommand{\cS}{\mathcal{S}}
\newcommand{\cR}{\mathcal{R}}
\newcommand{\cU}{\mathcal{U}}
\newcommand{\bcA}{\boldsymbol{\cA}}
\newcommand{\bcS}{\boldsymbol{\cS}}
\newcommand{\bvpi}{\boldsymbol{\varpi}}
\newcommand{\wcM}{\widetilde\cM}
\newcommand{\wcU}{\widetilde\cU}
\newcommand{\VT}{\Vert T_n \Vert}
\newcommand{\VcM}{\Vert \cM_n \Vert}
\newcommand{\VwcM}{\Vert \widetilde\cM_n \Vert}
\newcommand{\VcU}{\Vert \cU_n \Vert}
\newcommand{\VwcU}{\Vert \widetilde\cU_n \Vert}
\newcommand{\bi}{\boldsymbol{\infty}}
\newcommand{\bpi}{\boldsymbol{\pi}}
\newcommand{\bmu}{\boldsymbol{\mu}}
\newcommand{\N}{\mathbb{N}}
\newcommand{\Z}{\mathbb{Z}}
\newcommand{\C}{\mathbb{C}}
\newcommand{\ov}{\overline}
\newcommand{\wh}{\widehat}
\newcommand{\sn}{\sqrt{n}}

\begin{document}

%% Title, authors and addresses

\title{A polynomial solver based on numerical linear algebra : practical implementation and numerical experiments.}
\author{Jean-Paul Cardinal}

\documentclass{standalone}
% Preamble
\begin{document}

\begin{abstract}
Nous proposons un algorithme de calcul numérique des racines d'un système polynomial en intersection complète. Cet algorithme utilise les matrices de Bézout et ne fait appel qu'à des procédures d'algèbre linéaire. Il est possible d'éxécuter l'ensemble des calculs en arithmétique flottante. Une implémentation en Numpy/Octave/Sage est publiée sur le site \cite{jp_code}.
\end{abstract}

\end{document}

\documentclass{standalone}
% Preamble
\begin{document}

\section{Introduction}
Quelques mots d'introduction ici !!

\end{document}

\documentclass{standalone}
% Preamble
\begin{document}



\section{Cas univariable}
\label{univariable}

Rappelons quelques faits connus sur les polynômes à une variable.
Soit $f(x) = a_0x^d + \dots + a_{d-1}x + a_d$ un polynôme de la variable complexe $x$. Notons $<f(x)>$ l'idéal engendré par $f(x)$ dans l'anneau $\C[x]$ et $A_x = \C[x]/<f(x)>$  son algèbre quotient; de même nous noterons $A_y = \C[y]/<f(y)>$. Dorénavant $x$ désignera indifféremment la variable $x$, sa projection sur le quotient $A_x$ ou l'endomorphisme de multiplication par $x$ dans $A_x$. Une base du $\C$-espace vectoriel $A_x$ est la {\bf base des monômes} $\bold{x} = (1, x,\cdots, x^{d-1})$.

\subsection{Matrices des opérateurs de multiplication dans $A_x$}
L'opérateur de multiplication
$x : \left\vert
\begin{array}{c}
A \mapsto A \\
h \mapsto xh
\end{array}
\right.$
est un endomorphisme et se représente donc dans la base des monômes par une matrice $X$ de taille $d$, appelée {\bf matrice compagnon}
\begin{equation}
\label{compan}
X =
\begin{bmatrix}
	0 & \cdots & 0 & -a_d/a_0 \\
	1 & 0 & \cdots & -a_{d-1}/a_0 \\
	\vdots  & \ddots  & \ddots & \vdots  \\
	0 & \cdots & 1 & -a_1/a_0
\end{bmatrix}
\end{equation}

\begin{prop}
\label{compan2roots}
La matrice compagnon admet $f$ comme polynôme caractéristique et comme polynôme minimal, c'est-à-dire que $f(X) = 0$. De plus les racines du polynôme $f$ sont les valeurs propres de $X$, comptées avec les mêmes multiplicités.
\end{prop}

\begin{rem}
La proposition précédente fournit une méthode effective de calcul numérique des racines de $f$. En effet, $X$ est une matrice de Hessenberg, à laquelle on peut appliquer des techniques performantes de calcul de valeurs propres, comme la méthode du QR. Nous verrons à la section \ref{multivariable} que ces techniques peuvent aussi s'appliquer au cas d'un système multivariable en intersection complète.
\end{rem}
Plus généralement, pour tout élément $g\in A_x$, l'opérateur de multiplication
$g : \left\vert
\begin{array}{c}
A \mapsto A \\
h \mapsto gh
\end{array}
\right.$
est un endomorphisme, et se représente dans la base des monômes par une matrice que nous appelerons encore matrice compagnon de $g$ et qui se calcule à partir de $X$ très facilement. Considérons par exemple $g = x^2$. L'opérateur de multiplication par $x^2$ n'est autre que le carré de l'opérateur de multiplication par $x$; sa matrice est donc $X^2$. Donc d'une façon générale, nous avons
\begin{prop}
la matrice compagnon de $g$ est $g(X)$.
\end{prop}

\begin{rem}
Il faut noter que si $g_1, g_2$ sont deux représentants de $g$ on a $g_1(X) = g_2(X)$, ce qui définit $g(X)$ sans ambiguité, indépendamment du représentant de $g$ dans $A_x$.
\end{rem}

\subsection{Polynômes et matrices de Bezout}
\begin{defn}
\label{def_bez}
Introduisons une nouvelle variable $y$.
%et une famille $\bold{y} = (1, y, \cdots, y^{d-1})$, d'éléments de $A_y$.
Pour tout polynôme $g$, on définit le {\bf polynôme de Bezout $\delta(g)$} et la {\bf matrice de Bezout $B(g) = [b_{\alpha\beta}]$}  par les formules
\begin{equation}
\delta(g) = \dfrac{f(x)g(y)-f(y)g(x)}{x-y} = \sum_{\alpha,\beta = 0, \cdots, m-1} b_{\alpha\beta} x^\alpha y^\beta
\end{equation}
où $m$ désigne n'importe quel entier supérieur ou égal au maximum des degrés de $f$ et $g$.
\end{defn}

\begin{exmp}
Pour $f = x^2 - 3x + 2$, et $g = x^3$ on a les polynômes de Bezout $\delta(1) = -3 + x + y$ et $\delta(x^3) = -2x^2 - 2xy -2y^2 + 3x^2y + 3xy^2 -x^2y^2$ qui, représentés sous forme de tableaux, font apparaitre les matrices de Bezout $B(1)$ et $B(x^3)$
$$
\begin{array}{c|ccc}
\delta(1) & 1 & y & y^2\\
\hline
1 & -3 & 1 & 0\\
x & 1 & 0 & 0\\
x^2 & 0 & 0 & 0
\end{array}
\hspace{1cm}
\begin{array}{c|ccc}
\delta(x^3) & 1 & y & y^2\\
\hline
1 & 0 & 0 & -2\\
x & 0 & -2 & 3\\
x^2 & -2 & 3 & -1
\end{array}
$$
\end{exmp}

\begin{rem}
Le polynôme et la matrice de Bezout sont liés par l'égalité matricielle
\begin{equation}
	\label{pmB}
	\delta(g) = \bold{x} B(g) \bold{y}^T
\end{equation}
où $\bold{x} = (1, x,\cdots, x^{m-1})$ et $\bold{y} = (1, y,\cdots, y^{m-1})$ sont des vecteurs de monômes de $\C[x]$ et $\C[y]$. (Attention, on emploie encore ici la notation ${\bold x}$ pour un vecteur de $\C[x]^m$, notation qui était utilisée précédemment pour désigner la base des monômes ${\bold x}$ mais en pratique cette confusion n'est pas gênante).
\end{rem}
Considérons maintenant les produits vecteur-matrice $\bold{x}B(1)$ et $\bold{x}B(g)$. Ces deux familles sont constituées des colonnes de $B(1)$, resp. $B(g)$, vues comme des polynômes en $x$ exprimés dans la base des monômes. On peut aussi voir $\bold{x}B(1)$, resp. $\bold{x}B(g)$, comme la famille des coefficients du polynôme $\delta(1)$, resp. $\delta(g)$, considéré comme un polynôme en $y$ à coefficients dans $\C[x]$.


\begin{prop}
\label{relations_prop}
Soit $g$ un polynôme de $\C[x]$, et $m$ le maximum des degrés de $f$ et $g$. Si on écrit $B(1)$ et $B(g)$ dans le même système d'indice $\bold{x} = (1, x,\cdots, x^{m-1})$ et $\bold{y} = (1, y,\cdots, y^{m-1})$, alors
\begin{equation}
\label{relations}
	\bold{x}B(1)g = \bold{x}B(g)
\end{equation}
\end{prop}
\begin{proof}
Ecrivons
\begin{align} \nonumber
	\delta(g) = g(x)\dfrac{f(x)-f(y)}{x-y} - f(x)\dfrac{g(x)-g(y)}{x-y} \\ \nonumber
	\delta(g) = g(x)\delta(1) - f(x)\dfrac{g(x)-g(y)}{x-y}
\end{align}
Regardons cette dernière égalité comme une égalité entre polynômes en la variable $y$, à coefficients dans $\C[x]$. Si $h\in \C[x][y]$ et $\beta\in\N$ notons $h_\beta$ le coefficient de $y^\beta$ dans $h$. On a alors
$$\delta(g)_\beta = g(x)\delta(1)_\beta - f(x)(\dfrac{g(x)-g(y)}{x-y})_\beta $$
qui est une égalité entre éléments de $\C[x]$. En projetant sur $A_x$ on a
$\delta(g)_\beta = g(x)\delta(1)_\beta$
et comme ceci est vrai pour tout $\beta\in\N$, on obtient bien la relation~(\ref{relations}).
\end{proof}

\begin{rem}
En disant la  proposition autrement, chaque colonne de $B(1)$ donne, lorsqu'elle est multipliée par $g$ modulo $A_x$, la colonne de même indice de $B(g)$.
\end{rem}

\begin{exmp}
Reprenant l'exemple précédent, la Proposition \ref{relations_prop} dit que, modulo $A_x$, on a les égalités $(-3 + x)x^3 = -2x^2$, $(1)x^3 = -2x + 3x^2$, $(0)x^3 = -2 + 3x - x^2$, qui se vérifient facilement.
\end{exmp}

\begin{rem}
En considérant les lignes de $B(1), B(x)$ à la place des colonnes on aboutirait à une formule écrite en la variable $y$, identique à la formule~(\ref{relations}) car les matrices de Bezout sont ici symétriques, mais ce ne sera plus le cas en plusieurs variables.
\end{rem}

%\begin{rem}
%Les relations \ref{relations} sont d'un extrême importance pour nous et seront utilisées dans la construction du quotient dans le cas multivariable. Pour le dire à nouveau,, phénomène qui se traduit matriciellement par la formule de Barnett.
%\end{rem}

\subsection{Lien entre matrices de Bezout et matrices compagnon}
\label{Bar}
Particulièrement importantes sont les matrices de Bezout définies par $B(1)$ et $B(x)$
\begin{equation}
	\begin{array}{c|cccc}
		\delta(1) & 1 & y & \dots & y^{d-1} \\
		\hline
		1 & a_{d-1} & \ldots & \dots & a_0 \\
		x & a_{d-2} & \dots & a_0 & 0 \\
		\vdots & \vdots & \vdots & \vdots & \vdots \\
		x_{d-1} & a_0 & 0 & \ldots & 0 \\
	\end{array}
	\hspace{1.5cm}
	\begin{array}{c|cccc}
		\delta(x) & 1 & y & \dots & y^{d-1} \\
		\hline
		1 & -a_{d} & 0 & \dots & 0 \\
		x & 0 & a_{d-2} & \ldots & a_0 \\
		\vdots & \vdots & \vdots & \vdots & \vdots \\
		x_{d-1} & 0 & a_0 & \ldots & 0 \\
	\end{array}
\end{equation}
en effet nous avons le lien suivant entre matrice de Bezout et la matrice compagnon
\begin{prop}
\label{Barnett}
La matrice compagnon $X$ peut se calculer grâce à la {\bf formule de Barnett}
\cite{Barnett}
\begin{equation}
	B(x)B(1)^{-1} = X
\end{equation}
\end{prop}
\begin{proof}
Définissons deux nouvelles familles dans $A_x$ par
\begin{equation}
	\begin{array}{lll}
		\bold{x}B(1) & = & (a_{d-1} + a_{d-2}x + \cdots + a_0x^{d-1}, \cdots, a_1 + a_0x,  a_0).\\
		\bold{x}B(x) & = & (-a_d, a_{d-2}x + \cdots + a_0x^{d-1}, \cdots, a_0x)
	\end{array}
\end{equation}
et posons $\hat{\bold{x}} = \bold{x}B(1)$.
$B(1)$ étant inversible, la famille $\hat{\bold{x}}$ est une base de $A_x$ appellée {\bf base de Horner}.
D'après la proposition \ref{relations_prop} on a $\hat{\bold{x}}x = \bold{x}B(1)$. Par construction, les familles $\hat{\bold{x}}$ et $\hat{\bold{x}}x$ s'expriment dans la base $\bold{x}$ (des monômes) respectivement par les matrices $B(1)$ et $B(x)$.
La famille $\hat{\bold{x}}x$ s'exprime donc dans la base $\hat{\bold{x}}$ (de Horner) par la matrice $B(1)^{-1}B(x)$ ce qui veut dire que l'endomorphisme
$x : \left\vert
\begin{array}{c}
A \mapsto A \\
h \mapsto xh
\end{array}
\right.$ a pour matrice $B(1)^{-1}B(x)$ dans la base $\hat{\bold{x}}$
et pour matrice $B(1)(B(1)^{-1}B(x))B(1)^{-1} = B(x)B(1)^{-1}$ dans la base $\bold{x}$.
\end{proof}



\subsection{Formule de Barnett généralisée}
\label{Bar_gen}
La formule de Barnett a été écrite en considérant les matrices de Bezout des polynômes $1$ et $x$.
Si on considère un polynôme quelconque $g$ de $\C[x]$ et $B(g)$ sa matrice de Bezout il serait naturel d'avoir entre les matrices de Bezout $B(1)$ et $B(g)$ la relation suivante, que nous appellerons {\bf formule de Barnett généralisée}
\begin{equation}
	\label{BG}
	B(g)B(1)^{-1} = g(X)
\end{equation}
On montre facilement la formule~(\ref{BG}) lorsque le degré de g est inférieur ou égal à $d$, c'est-à-dire lorsque $B(1)$ et $B(g)$ sont de même taille. Par exemple pour $f = x^2 - 3x + 2$, $d = 2$, on a
$$
\begin{array}{c|cc}
	\delta(1) & 1 & y \\
	\hline
	1 & -3 & 1 \\
	x & 1 & 0
\end{array}
\hspace{1cm}
\begin{array}{c|cc}
	\delta(x) & 1 & y \\
	\hline
	1 & -2 & 1 \\
	x & 1 & 0
\end{array}
\hspace{1cm}
\begin{array}{c|cc}
	\delta(x^2) & 1 & y \\
	\hline
	1 & 0 & -2 \\
	x & -2 & 3
\end{array}
$$
\begin{equation}
	B(x)B(1)^{-1} =
	\begin{bmatrix}
		0 & -2 \\
		1 & 3
	\end{bmatrix}
	= X
	\hspace{1cm}
	B(x^2)B(1)^{-1} =
	\begin{bmatrix}
		-3 & -6 \\
		2 & 7
	\end{bmatrix}
	= X^2
\end{equation}
ce qui confirme bien la formule~(\ref{BG}).
Si par contre le degré de $g$ est supérieur à $d$, alors $B(g)$ et $B(1)$ ne sont plus de la même taille et une opération telle que $B(g)B(1)^{-1}$ n'a plus de sens. Une première idée est de réécrire les deux matrices de Bezout dans le même système d'indices, à savoir
$\bold{x} = (1, x,\cdots, x^{m-1})$ et $\bold{y} = (1, y,\cdots, y^{m-1})$, $m$ étant le degré de $g$. Par exemple en choisissant $f$ comme ci-dessus et $g = x^3$ on aurait
$$
\begin{array}{c|ccc}
\delta(1) & 1 & y & y^2\\
\hline
1 & -3 & 1 & 0\\
x & 1 & 0 & 0\\
x^2 & 0 & 0 & 0
\end{array}
\hspace{1cm}
\begin{array}{c|ccc}
\delta(x^3) & 1 & y & y^2\\
\hline
1 & 0 & 0 & -2\\
x & 0 & -2 & 3\\
x^2 & -2 & 3 & -1
\end{array}
$$
mais alors $B(1)$ n'est plus inversible. Nous allons cependant montrer, grâce aux relations~(\ref{relations}), que si {\bf on projette les deux polynômes de Bezout sur le quotient $A_x$} alors les matrices $B(g)$ et $B(1)$ sont redimensionnées à la même taille, $B(1)$ est inversible et la formule~(\ref{BG}) s'applique. Illustrons le procédé sur l'exemple ci-dessus. Puisque $B(1)$ n'est pas inversible, on peut trouver une combinaison linéaire de colonnes qui s'annule, ici c'est la troisième colonne qui est nulle. En la multipliant par $x^3$, et en appliquant les relations~(\ref{relations}), on obtient que la troisième colonne de $B(x^3)$ est aussi nulle. Mais cette colonne vaut $-2 + 3x - x^2$ ce qui entraine que, dans le quotient,
$-2 + 3x - x^2 = 0$ (ce n'est pas une surprise car ce dernier polynôme n'est autre que $-f$; ceci est dû au fait que l'exemple choisi est particulièrement simple, mais nous verrons dans le cas multidimensionnel que les relations nulles dans le quotient ainsi générées sont loin d'être triviales).

En vue d'automatiser les calculs, traduisons le procédé précédent en termes d'algèbre matricielle.
Toujours sur le même exemple
\begin{align} \nonumber %%%%%%%%%%%%%%
	\delta(x^3) &=
	\begin{bmatrix}
			1 & x & x^2
	\end{bmatrix}
	\begin{bmatrix}
		0 & 0 & -2 \\
		0 & -2 & 3 \\
		-2 & 3 & -1
	\end{bmatrix}
	\begin{bmatrix}
		1 \\
		y \\
		y^2
	\end{bmatrix} \\ \nonumber %%%%%%%%%%%%%%
	\delta(x^3) &=
	\begin{bmatrix}
		1 & x & x^2
	\end{bmatrix}
	\begin{bmatrix}
		1 & 0 & 2 \\
		0 & 1 & -3 \\
		0 & 0 & 1
	\end{bmatrix}
	\begin{bmatrix}
		1 & 0 & -2 \\
		0 & 1 & 3 \\
		0 & 0 & 1
	\end{bmatrix}
	\begin{bmatrix}
		0 & 0 & -2 \\
		0 & -2 & 3 \\
		-2 & 3 & -1
	\end{bmatrix}
	\begin{bmatrix}
		1 \\
		y \\
		y^2
	\end{bmatrix} \\ \nonumber %%%%%%%%%%%%%%%
	\delta(x^3) &=
	\begin{bmatrix}
			1 & x & 2 - 3x + x^2
	\end{bmatrix}
	\begin{bmatrix}
		4 & -6 & 0 \\
		-6 & 7 & 0 \\
		-2 & 3 & -1
	\end{bmatrix}
	\begin{bmatrix}
		1 \\
		y \\
		y^2
	\end{bmatrix} \\ \nonumber %%%%%%%%%%%%%%
\end{align}
En résumé nous multiplions le vecteur d'indices
$\begin{bmatrix}
	1 & x & x^2
\end{bmatrix}$ à droite par la transformation de Gauss
$P =
\begin{bmatrix}
	1 & 0 & 2 \\
	0 & 1 & -3 \\
	0 & 0 & 1
\end{bmatrix}$
et les deux matrices de Bezout $B(1)$ et $B(g)$ à gauche par $P^{-1}$. Les polynômes de Bezout, écrits sous forme de tableaux, deviennent alors
$$
\begin{array}{c|ccc}
	\delta(1) & 1 & y & y^2\\
	\hline
	1 & -3 & 1 & 0\\
	x & 1 & 0 & 0\\
	2 - 3x + x^2 & 0 & 0 & 0
\end{array}
\hspace{1cm}
\begin{array}{c|ccc}
	\delta(x^3) & 1 & y & y^2\\
	\hline
	1 & 4 & -6 & 0 \\
	x & -6 & 7 & 0 \\
	2 - 3x + x^2 & -2 & 3 & -1
\end{array}
$$
Ce que disent les relations~(\ref{relations}) c'est que la troisième colonne de $B(x^3)$ est nulle dans le quotient $A_x$, c'est à-dire $-2 + 3x - x^2 = 0$, (on reconnait l'égalité $-f = 0$). On a donc
$$
\delta(1) = \begin{bmatrix}
	1 & x
\end{bmatrix}
\begin{bmatrix}
	-3 & 1 \\
	1 & 0
\end{bmatrix}
\begin{bmatrix}
	1 \\
	y
\end{bmatrix}$$
$$\delta(x^3) = \begin{bmatrix}
	1 & x
\end{bmatrix}
\begin{bmatrix}
	4 & -6 \\
	-6 & 7
\end{bmatrix}
\begin{bmatrix}
	1 \\
	y
\end{bmatrix} + (2 - 3x + x^2)(-2 + 3y - y^2)$$
puis, en projetant $\delta(1), \delta(g)$ sur $A_x \otimes A_y$
$$
\begin{array}{c|cc}
	\delta(1) & 1 & y \\
	\hline
	1 & -3 & 1 \\
	x & 1 & 0
\end{array}
\hspace{1cm}
\begin{array}{c|cc}
	\delta(x^3) & 1 & y \\
	\hline
	1 & 4 & -6  \\
	x & -6 & 7
\end{array}
$$
Nous avons bien obtenu des matrices de Bezout de même taille, avec $B(1)$ inversible. Formons alors le quotient
\begin{equation}
	B(x^3)B(1)^{-1} =
	\begin{bmatrix}
		-6 & -14 \\
		7 & 15
	\end{bmatrix}
	= X^3
\end{equation}
ce qui est bien conforme à la formule de Barnett généralisée.

\begin{rem}
On peut remplacer la matrice de Gauss par toute matrice permettant de transformer une colonne donnée en une colonne possédant un seul élément non nul, comme par exemple une matrice orthogonale de Householder. C'est le choix qui sera fait dans l'implémentation en Octave proposée en annexe.

\end{rem}

\end{document}

\documentclass{standalone}
% Preamble
\begin{document}


\section{Cas multivariable}
\label{multivariable}

Dans le cas univariable, examiné à la section précédente, la structure de $A_x$ se compose d'une part d'une base, en l'occurence la base des monômes, d'autre part de la matrice compagnon, exprimant l'endomorphisme de multiplication par $x$ dans la base. Ces deux éléments peuvent être obtenus soit directement par lecture des coefficients de $f$, soit à partir des matrices de Bezout $B(1), B(x)$. \\
Dans le le cas multivariable, que nous allons développer dans cette section, ni une base ni les matrices compagnon
(matrices des opérateurs
$x_j : \left\vert
\begin{array}{c}
A \mapsto A \\
h \mapsto x_jh
\end{array}
\right.$ dans la base) ne sont visibles directement sur les coefficients des polynômes de départ. En revanche nous allons montrer comment construire des matrices de Bezout $B(1), B(x_1), \cdots, B(x_n)$ à partir desquelles on peut obtenir une base et les matrices compagnon $X_j$ associées à la base obtenue. Commençons par fixer le cadre de travail.
Pour $n$ polynômes $f_1,\cdots, f_n$ en les variables complexes $x_1,\cdots, x_n$ considérons :
\begin{itemize}
\item $\C[x] = \C[x_1,\cdots, x_n]$ l'anneau des polynômes en les variables $x = x_1,\cdots, x_n$
\item $<f> = <f_1,\cdots, f_n>$ l'idéal généré
\item $V(f) = \{x \in \C^n : f(x) = 0\}$ la variété associée à $<f>$
\item $A_x = \C[x]/<f>$ l'algèbre quotient
\end{itemize}
Nous supposerons dorénavant que l'idéal $<f>$ est {\bf zéro-dimensionel}, c'est-à-dire que $V(I)$ est fini ou, de façon équivalente \cite[p.~234]{clo}, que $A_x$ est de {\bf dimension finie} en tant qu'espace vectoriel sur $\C$. Ceci est bien sûr toujours le cas lorsque $n = 1$.

\subsection{Construction des polynômes et des matrices de Bezout}

\subsubsection{Extension de la Définition \ref{def_bez} au cas multivariable}
\label{def_bez_multi}

\begin{defn}
Soit $x^\gamma = x_1^{\gamma_1}\cdots x_n^{\gamma_n} \in \C[x]$ un monôme.
Introduisons un nouveau jeu de variables $y = y_1,\cdots, y_n$ et considérons, pour tous $i, j = 1\cdots n$, le rapport
\begin{equation}
\label{finite_diff}
\delta_{i,j}(x^\gamma) = \dfrac{y_j^{\gamma_j}f_i(y_1,\cdots, y_{j-1},x_j,\cdots,x_n) - x_j^{\gamma_j}f_i(y_1,\cdots,y_j,x_{j+1},\cdots,x_n)}{x_j - y_j}
\end{equation}
qui est un polynôme en les variables $x, y$. Nous obtenons une matrice de différences finies $\Delta(x^\gamma) = (\delta_{ij}(x^\gamma))_{ij}$, qui est à la matrice jacobienne ce que le taux d'accroissement est à la dérivée.
Le {\bf polynôme de Bezout} du monôme $x^\gamma$ est par définition
\begin{equation}
	\delta(x^\gamma) = det(\Delta(x^\gamma))
\end{equation}
qui est un élément de $\C[x, y]$. Pour un polynôme général $g = \sum_\gamma g_\gamma x^\gamma \in \C[x]$, on étend la définition précédente par linéarité $\delta(g) = \sum_\gamma g_\gamma \delta(x^\gamma)$.
En développant $\delta(g) = \sum_{\alpha,\beta} b_{\alpha\beta} x^\alpha y^\beta$ comme une somme de monômes de $\C[x, y]$, et en notant $\bold{x}$ et $\bold{y}$ les familles de tous les monômes en $x$ et $y$ apparaissant dans ce développement, nous définissons la {\bf matrice de Bezout} $B(g) = [b_{\alpha\beta}]$, c'est à dire que l'on a la relation suivante, similaire à~(\ref{pmB}), entre polynôme et matrice de Bezout
\begin{equation}
	\delta(g) = \bold{x} B(g) \bold{y}^T
\end{equation}
\end{defn}

\begin{exmp}
\label{ex_bez_multi}
Fixons $n = 2$ et considérons $f_1 = x_1^2 + x_1x_2^2 - 1, f_2 = x_1^2x_2 + x_1$.
Nous allons calculer les matrices de Bezout $B(1), B(x_1), B(x_2)$,  qui vont servir à la construction des matrices compagnon $X_1, X_2$, comme nous le verrons plus loin. Pour commencer, calculons à partir des formules (\ref{finite_diff})
\begin{align}
\Delta(1) &=
\begin{pmatrix}
x_1 + x_2^2 + y_1 & x_2y_1 + y_1y_2 \\
1 + x_1x_2 + x_2y_1 & y_1^2
\end{pmatrix} \nonumber  \\
\Delta(x_1) &=
\begin{pmatrix}
1 + x_1y_1 & x_2y_1 + y_1y_2 \\
1 + x_1x_2 + x_2y_1 & y_1^2
\end{pmatrix} \nonumber  \\
\Delta(x_2) &=
\begin{pmatrix}
x_1 + x_2^2 + y_1 & 1 - y_1^2 + x_2y_1y_2 \\
1 + x_1x_2 + x_2y_1  & -y_1
\end{pmatrix} \nonumber
\end{align}
dont le déterminant fournit les polynômes de Bezout
\begin{align}
\delta(1) &= -x_2y_1 - x_1x_2^2y_1 + x_1y_1^2 + y_1^3 - y_1y_2 - x_1x_2y_1y_2 - x_2y_1^2y_2 \nonumber \\
\delta(x_1) &=  y_1^2 - x_1x_2^2y_1^2 + x_1y_1^3 - x_1x_2y_1^2y_2 \nonumber \\
\delta(x_2) &= -1 - x_1x_2 - x_1y_1 -x_2y_1 - x_2^2y_1 + x_1x_2y_1^2 + x_2y_1^3 - x_2y_1y_2 - x_1x_2^2y_1y_2 - x_2^2y_1^2y_2\nonumber
\end{align}
Les familles de mônomes apparaissant dans ces polynômes sont
$\bold{x} = (1, x_2, x_2^2, x_1, x_1x_2, x_1x_2^2)$ et $\bold{y} = (1, y_1, y_1y_2, y_1^2, y_1^2y_2, y_1^3)$.
Les polynômes de Bezout s'écrivent sous forme de tableaux faisant apparaitre les matrices de Bezout
$$\begin{array}{c|cccccc}
	\delta(1) & 1 & y_1 & y_1y_2 & y_1^2 & y_1^2y_2 & y_1^3 \\
	\hline
	1 &  &  & -1 &  &  & 1\\
	x_2 &  & -1 &  &  & -1 & \\
	x_2^2 &  &  &  &  &  & \\
	x_1 &  &  &  & 1 &  & \\
	x_1x_2 &  &  & -1 &  &  & \\
	x_1x_2^2 &  & -1 &  &  &  &
\end{array}$$

$$\begin{array}{c|cccccc}
	\delta(x_1) & 1 & y_1 & y_1y_2 & y_1^2 & y_1^2y_2 & y_1^3 \\
	\hline
	1 &  &  &  & 1 &  & \\
	x_2 &  &  &  &  &  & \\
	x_2^2 &  &  &  &  &  & \\
	x_1 &  &  &  &  &  & 1\\
	x_1x_2 &  &  &  &  & -1 & \\
	x_1x_2^2 &  &  &  & -1 &  &
\end{array}
\hspace{0.2cm}
\begin{array}{c|cccccc}
	\delta(x_2) & 1 & y_1 & y_1y_2 & y_1^2 & y_1^2y_2 & y_1^3 \\
	\hline
	1 & -1 &  &  &  &  & \\
	x_2 &  & -1 & -1 &  &  & 1\\
	x_2^2 &  & -1 &  &  & -1 & \\
	x_1 &  & -1 &  &  &  & \\
	x_1x_2 & -1 &  &  & 1 &  & \\
	x_1x_2^2 &  &  & -1 &  &  &
\end{array}$$

\end{exmp}
\begin{rem}
Ici, contrairement au cas univariable, les listes $\bold{x}$ et $\bold{y}$ ne sont pas des bases de $A$. Nous verrons plus loin qu'elles sont cependant génératrices et comment on peut en extraire des bases.
\end{rem}

\subsubsection{Calcul effectif des matrices de Bezout}
Dans l'exemple précédent, les polynômes de Bezout s'obtiennent en calculant le déterminant des matrices $\Delta(1), \Delta(x_1), \Delta(x_2)$, qui sont de taille $2$ et dont les coefficients sont des polynômes en $x_1, x_2$. Si le nombre de variables $n$ ou le degré des polynômes $f_i$ augmentaient alors ce calcul pourrait devenir difficile car les coefficients des matrices $\Delta(x_j)$ ne sont pas numériques et on ne peut donc pas appliquer la méthode du pivot de Gauss. Un moyen de contourner cette difficulté est de procéder par évaluation-interpolation :
\begin{enumerate}
\item
on estime à priori l'ensemble des monômes qui vont apparaitre dans le résultat $\delta(x^\gamma)$
\item
on évalue $\Delta(x^\gamma)$ sur un ensemble adéquat $U \times V$ de multi-points de Fourier $u = (u_1,\cdots, u_n) \in U$ et $v = (v_1,\cdots, v_n) \in V$
\item
pour chaque point $(u, v) \in U\times V$ on calcule le déterminant numérique de $\Delta(x^\gamma)(u, v)$ par la méthode du pivot de Gauss.
\item
pour finir on interpole l'ensemble des valeurs obtenues par le polynôme cherché $\delta(x^\gamma)$.
\end{enumerate}

Pour implémenter cet algorithm concrètement on doit préciser l'ensemble des monômes de $\delta(x^\gamma)$ ainsi que les points de Fourier utilisés pour l'évaluation de $\delta(x^\gamma)$. Prenons l'exemple d'un système polynomial $\left<f\right>$ de multidegré $(d_1, \cdots, d_n)$, c'est-à-dire que pour tous $i, j = 1..n$ le degré de $f_i$ en la variable $x_j$ est inférieur ou égal à $d_j$. Fixons alors un entier $k$ compris entre $0$ et $n$ et adoptons la convention que $x_0 = 1$. On voit alors facilement que $\delta(x_k)$, polynôme en $x, y$, est de multidegré $(d_1, 2d_2, \cdots, nd_n)$ en $x$ et de multidegré $(nd_1, (n-1)d_2, \cdots, d_n)$ en $y$.
 Pour l'évaluation de $\delta(x_k)$ aux points de Fourier $(u, v) \in U\times V$ nous choisissons donc $U = \prod_{j=1..n} U_j, V = \prod_{j=1..n} V_j$ o\`u $U_j$ est l'ensemble des racines complexes de $X^{jd_j} - 1$ et $V_j$ est l'ensemble des racines complexes de $X^{(n-j+1)d_j} - \theta_j$ et o\`u $\theta_j$ est choisi de façon que $U_j$ et $V_j$ soient disjoints, afin que le dénominateur ne s'annule jamais dans la formule (\ref{finite_diff}). Ceci est réalisé par exemple si $\theta_j = e^{i\pi/j}$.

Les considérations précédentes nous permettent d'écrire l'algorithme :

\begin{algorithm}
\caption{Construction des points de Fourier}\label{fourierPoints}
\begin{algorithmic}
\Function{fourierPoints}{$d$} \Comment{multidegré $d = (d_1,\cdots,d_n)$}
\For{$j=1..n$} \Comment{construction des facteurs $U_j, V_j$}
\State $U_j \gets$ ensemble des racines de $X^{jd_j}-1$
\State $V_j \gets$ ensemble des racines de $X^{(n-j+1)d_j}-e^{i\pi/j}$
\EndFor
\State $U \gets \prod_{j=1..n}U_j$
\State $V \gets \prod_{j=1..n}V_j$
\State \textbf{return} $U, V$
\EndFunction
\end{algorithmic}
\end{algorithm}

Les ensembles de points de Fourier $U$ et $V$ peuvent maintenant être utilisés pour construire la matrice d'évaluation du polynôme de Bezout :
\begin{algorithm}[H]
\caption{Construction de la matrice d'évaluation du polynôme de Bezout}\label{evaluation}
\begin{algorithmic}
\Function{evaluation}{$f, k$} \Comment{$f = (f_1,\cdots,f_n)$ système polynomial}\\
\Comment{ $k$ entier, $0\le k \le n$}
\State $U, V \gets \textsc{fourierPoints}(d)$
\State $D \gets \prod_{j=1..n}jd_j$
\State $C^{(k)} \gets \textsc{zeros}(D, D)$
\For{$(u, v) \in U\times V$}
      \State $\Delta \gets \textsc{zeros}(n, n)$
   		\For{$i, j=1..n$}
      		\State $\Delta_{i,j} \gets \delta_{i,j}(x_k)(u, v)$ \Comment{$\delta_{i,j}(x_k)$ défini à la formule (\ref{finite_diff})}
   		\EndFor
		\State $C^{(k)}_{u, v} \gets \textsc{det}(\Delta)$
	\EndFor
\State \textbf{return} $C^{(k)}$
\EndFunction
\end{algorithmic}
\end{algorithm}

Rappelons que la matrice de Bezout $B^{(k)} = \left[b^{(k)}_{\alpha\beta}\right]_{\alpha\beta}$ est définie par $\delta^{(k)}(x, y) = \sum_{\alpha,\beta} b^{(k)}_{\alpha\beta} x^\alpha y^\beta$. On a donc $C^{(k)}_{u,v} = \delta^{(k)}(u, v) = \sum_{\alpha,\beta} b^{(k)}_{\alpha\beta} u^\alpha v^\beta$, ce qui s'écrit comme produit de matrices
$\left[C^{(k)}_{u,v}\right]_{u,v} = \left[u^\alpha\right]_{u,\alpha} \left[b^{(k)}_{\alpha,\beta}\right]_{\alpha, \beta} \left[v^\beta\right]_{v, \beta}^T$.
Définissons alors les matrices de Fourier $F_u, F_v$ par $F_u = \left[ u^\alpha \right]_{u, \alpha}$ et $F_v = \left[ v^\beta \right]_{v, \beta}$. On obtient alors la relation d'évaluation-interpolation entre les matrices $B^{(k)}$ et $C^{(k)}$
\begin{equation}
C^{(k)} = F_uB^{(k)} F_v^T
\end{equation}


\begin{algorithm}
\caption{Construction des matrices de Fourier}\label{fourierPoints}
\begin{algorithmic}
\Function{fourierMatrices}{$d$} \Comment{multidegré $d = (d_1,\cdots,d_n)$}
	\State $U, V \gets \textsc{fourierPoints}(d)$
	\For{$j=1..n$}
		\State $D_j \gets \prod_{j=1..n}\left[0..jd_j-1\right]$ \Comment{intervalle d'entiers de $0$ à $jd_j-1$}
		\State $\bar{D_j} \gets \prod_{j=1..n}\left[0..(n-j+1)d_j-1\right]$
	\EndFor
	\State ${D} \gets \prod_{j=1..n}D_j$ \Comment{produit cartésien}
	\State $\bar{D} \gets \prod_{j=1..n}\bar{D_j}$
	\For{$(u, \alpha) \in U\times D$}
		\State $F_{u,\alpha} \gets u^\alpha$ \Comment{$u^\alpha = \prod_{j=1..n}u_j^{\alpha_j}$}
	\EndFor
	\For{$(v, \beta) \in V\times \bar{D}$}
		\State $G_{v,\beta} \gets v^\beta$
	\EndFor
	\State \textbf{return} $F, G$
\EndFunction
\end{algorithmic}
\end{algorithm}

\begin{algorithm}[H]
\caption{Construction de la matrice d'évaluation du polynôme de Bezout}\label{evaluation}
\begin{algorithmic}
\Function{evaluation}{$f, k$} \Comment{$f = (f_1,\cdots,f_n)$ système polynomial}\\
\Comment{ $k$ entier, $0\le k \le n$}
\State $U, V \gets \textsc{fourierPoints}(d)$
\State $D \gets \prod_{j=1..n}jd_j$
\State $C^{(k)} \gets \textsc{zeros}(D, D)$
\For{$(u, v) \in U\times V$}
      \State $\Delta \gets \textsc{zeros}(n, n)$
   		\For{$i, j=1..n$}
      		\State $\Delta_{i,j} \gets \delta_{i,j}(x_k)(u, v)$ \Comment{$\delta_{i,j}(x_k)$ défini à la formule (\ref{finite_diff})}
   		\EndFor
		\State $C^{(k)}_{u, v} \gets \textsc{det}(\Delta)$
	\EndFor
\State \textbf{return} $C^{(k)}$
\EndFunction
\end{algorithmic}
\end{algorithm}



\subsection{Formules de Barnett et structure de l'algèbre quotient.}
Rappelons que dans notre hypothèse où l'idéal est zéro-dimensionel, la dimension de l'algèbre quotient $A = \C[\bold{x}]/<f>$ est finie, ce qui assure l'existence d'une base et de matrices compagnon $X_1,\cdots, X_n$ (matrices des opérateurs de multiplication par les variables dans la base considérée). Nous allons montrer que le procédé mis en oeuvre dans la section \ref{Bar_gen}, consistant en manipulations sur les colonnes des matrices de Bezout $B_0, \cdots, B_n $, permettra de construire une base de $A_x$ ainsi que les matrices compagnons associées.\\
Rappelons tout d'abord un certain nombre de propriétés algébriques du polynôme $\delta(1)$ et des matrices de Bezout $B_j$.

\subsubsection{Propriétés algébriques du polynôme $\delta(1)$ et de la matrice $B_0$}
Les propriétés qui suivent sont de nature algébrique et seront données sans démonstration. Le lecteur intéressé pourra consulter les détails dans \cite{jpc, CM}. Comme dans la proposition \ref{Barnett}, définissons de nouvelles familles dans $A_x$ par
\begin{equation}
		\hat{\bold{x}}_j  =  \bold{x}B_j, \quad j=0\cdots n
\end{equation}

\begin{exmp}
En reprenant l'exemple précédent nous avons
\begin{equation}
	\begin{array}{lll}
		\hat{\bold{x}}_0 & = & (0, -x_2 - x_1x_2^2, -1 - x_1x_2, x_1, -x_2, 1) \\
		\hat{\bold{x}}_1 & = & (0, 0, 0, -1 - x_2^2, -x_1x_2, x_1) \\
		\hat{\bold{x}}_2 & = & (-1 - x_1x_2, -x_2 - x_2^2 - x_1, - x_2 - x_1x_2^2, x_1x_2, -x_2^2, x_2)
	\end{array}
\end{equation}
\end{exmp}

\begin{prop}
\label{xj}
Pour tout $j=1\cdots n$ on a
\begin{equation}
    \hat{\bold{x}}_0x_j = \hat{\bold{x}}_j
\end{equation}
\end{prop}

Les relations ci-dessus sont faciles à vérifier sur l'exemple précédent. Jusqu'à maintenant les cas univariable et multivariable sont très similaires, sauf sur un point : dans le cas multivariable les familles $\bold{x}_0$ et $\hat{\bold{x}}_0$ ne sont plus nécessairement des bases de $A_x$. On a cependant la propriété suivante

\begin{prop}
Chacune des familles $\bold{x}$ et $\hat{\bold{x}}$ est génératrice dans $A_x$.
\end{prop}




%\begin{figure}[!h]
%	\includegraphics[scale=0.3]{relations_svd}
%\end{figure}

\subsubsection{Processus de réduction}
La proposition précédente fournit un début de structure de l'algèbre $A_x$. Pour l'instant nous avons une famille génératrice $\bold{x}$ de $A_x$ ainsi que des matrices de Bezout $B(x_j)$.
Nous allons montrer comment, en appliquant le procédé matriciel décrit dans la section \ref{Bar_gen} à la famille génératrice $\bold{x}$ et aux matrices de Bezout $B_j$, on peut fabriquer une base de $A_x$ et des matrices compagnon $X_j$.
Illustrons les calculs à partir de l'exemple \ref{ex_bez_multi}.\\
Le rang de $B_0$ est $5$. La première colonne de $B_0$ est nulle mais celle de $B_2$ ne l'est pas, ce qui fournit la relation dans le quotient $1 + x_1x_2 = 0$.
Multiplions $\bold{x}$ à droite par la matrice de Gauss $P$ dont la $5$ième colonne vaut $(1, 0, 0, 0, 1, 0)^{T}$, et multiplions les matrices de Bezout à gauche par $P^{-1}$, ce qui revient à soustraire la $5$ième ligne à la première.
Les polynômes de Bezout se reécrivent\\

\resizebox{\linewidth}{!}{$
\begin{array}{c|cccccc}
	\delta(1) & 1 & y_1 & y_1y_2 & y_1^2 & y_1^2y_2 & y_1^3 \\
	\hline
	1 &  &  &  &  &  & 1\\
	x_2 &  & -1 &  &  & -1 & \\
	x_2^2 &  &  &  &  &  & \\
	x_1 &  &  &  & 1 &  & \\
	1+x_1x_2 &  &  & -1 &  &  & \\
	x_1x_2^2 &  & -1 &  &  &  &
\end{array}
\hspace{0.2cm}
\begin{array}{c|cccccc}
	\delta(x_1) & 1 & y_1 & y_1y_2 & y_1^2 & y_1^2y_2 & y_1^3 \\
	\hline
	1 &  &  &  & 1 & 1 & \\
	x_2 &  &  &  &  &  & \\
	x_2^2 &  &  &  &  &  & \\
	x_1 &  &  &  &  &  & 1\\
	1+x_1x_2 &  &  &  &  & -1 & \\
	x_1x_2^2 &  &  &  & -1 &  &
\end{array}
\hspace{0.2cm}
\begin{array}{c|cccccc}
	\delta(x_2) & 1 & y_1 & y_1y_2 & y_1^2 & y_1^2y_2 & y_1^3 \\
	\hline
	1 &  &  &  & -1 &  & \\
	x_2 &  & -1 & -1 &  &  & 1\\
	x_2^2 &  & -1 &  &  & -1 & \\
	x_1 &  & -1 &  &  &  & \\
	1+x_1x_2 & -1 &  &  & 1 &  & \\
	x_1x_2^2 &  &  & -1 &  &  &
\end{array}
$}
La première colonne de $B_2$ contient maintenant un seul coefficient non nul, indexé par $1 + x_1x_2$, on peut donc, en projetant les trois bezoutiens sur $A_x$, supprimer la première colonne et la cinquième ligne dans les trois matrices. On obtient\\

\resizebox{\linewidth}{!}{$
\begin{array}{c|ccccc}
	\delta(1) & y_1 & y_1y_2 & y_1^2 & y_1^2y_2 & y_1^3 \\
	\hline
	1  &  &  &  &  & 1 \\
	x_2  & -1 &  &  & -1 & \\
	x_2^2  &  &  &  &  & \\
	x_1  &  &  & 1 &  & \\
	x_1x_2^2  & -1 &  &  &  &
\end{array}
\hspace{0.2cm}
\begin{array}{c|ccccc}
	\delta(x_1)  & y_1 & y_1y_2 & y_1^2 & y_1^2y_2 & y_1^3 \\
	\hline
	1  &  &  & 1 & 1 & \\
	x_2  &  &  &  &  & \\
	x_2^2  &  &  &  &  & \\
	x_1  &  &  &  &  & 1 \\
	x_1x_2^2  &  &  & -1 &  &
\end{array}
\hspace{0.2cm}
\begin{array}{c|ccccc}
	\delta(x_2) & y_1 & y_1y_2 & y_1^2 & y_1^2y_2 & y_1^3 \\
	\hline
	1  &  &  & -1 &  & \\
	x_2  & -1 & -1 &  &  & 1 \\
	x_2^2  & -1 &  &  & -1 & \\
	x_1  & -1 &  &  &  & \\
	x_1x_2^2 &  & -1 &  &  &
\end{array}
$}\\

Maintenant, la deuxième colonne de $B_0$ est nulle, celle de $B_2$ ne l'est pas. La relation est $x_2 + x_1x_2^{2} = 0$. La matrice $P$ est définie par sa cinquième colonne $(0, 1, 0, 0, 1)^{T}$. Le vecteur $\bold{x}$ devient $(1, x_2, x_2^{2}, x_1, x_2 + x_1x_2^{2})$. On soustrait la cinquième ligne à la deuxième. Les bezoutiens se reécrivent\\

\resizebox{\linewidth}{!}{$
\begin{array}{c|ccccc}
	\delta(1) & y_1 & y_1y_2 & y_1^2 & y_1^2y_2 & y_1^3 \\
	\hline
	1  &  &  &  &  & 1 \\
	x_2  &  &  &  & -1 & \\
	x_2^2  &  &  &  &  & \\
	x_1  &  &  & 1 &  & \\
	x_2 + x_1x_2^2  & -1 &  &  &  &
\end{array}
\hspace{0.2cm}
\begin{array}{c|ccccc}
	\delta(x_1)  & y_1 & y_1y_2 & y_1^2 & y_1^2y_2 & y_1^3 \\
	\hline
	1  &  &  & 1 & 1 & \\
	x_2  &  &  & 1 &  & \\
	x_2^2  &  &  &  &  & \\
	x_1  &  &  &  &  & 1 \\
	x_2 + x_1x_2^2  &  &  & -1 &  &
\end{array}
\hspace{0.2cm}
\begin{array}{c|ccccc}
	\delta(x_2) & y_1 & y_1y_2 & y_1^2 & y_1^2y_2 & y_1^3 \\
	\hline
	1  &  &  & -1 &  & \\
	x_2  & -1 &  &  &  & 1 \\
	x_2^2  & -1 &  &  & -1 & \\
	x_1  & -1 &  &  &  & \\
	x_2 + x_1x_2^2 &  & -1 &  &  &
\end{array}
$}\\

La deuxième colonne de $B_2$ contient un seul coefficient non nul, en cinquième ligne, on peut donc supprimer les deuxièmes colonnes et les cinquièmes lignes. On obtient\\

$$
\begin{array}{c|cccc}
	\delta(1) & y_1 & y_1^2 & y_1^2y_2 & y_1^3 \\
	\hline
	1  &   &  &  & 1 \\
	x_2  &  &  & -1 & \\
	x_2^2  &  &  &  & \\
	x_1  &  & 1 &  &
\end{array}
\hspace{0.2cm}
\begin{array}{c|cccc}
	\delta(x_1)  & y_1 & y_1^2 & y_1^2y_2 & y_1^3 \\
	\hline
	1  &  & 1 & 1 & \\
	x_2  &  & 1 &  & \\
	x_2^2  &  &  &  & \\
	x_1  &  &  &  & 1
\end{array}
\hspace{0.2cm}
\begin{array}{c|cccc}
	\delta(x_2) & y_1 & y_1^2 & y_1^2y_2 & y_1^3 \\
	\hline
	1  &  & -1 &  & \\
	x_2  & -1 &  &  & 1 \\
	x_2^2  & -1 &  & -1 & \\
	x_1  & -1 &  &  &
\end{array}
$$\\

Maintenant, la première colonne de $B_0$ est nulle, celle de $B_2$ ne l'est pas. La relation est $x_2 + x_2^{2} + x_1 = 0$. La matrice $P$ est définie par sa quatrième colonne $(0, 1, 1, 1)^{T}$. Le vecteur $\bold{x}$ devient $(1, x_2, x_2^{2},  x_2 + x_2^{2} + x_1)$. On soustrait la quatrième ligne à la deuxième et à la troisième. Les bezoutiens se reécrivent\\

\resizebox{\linewidth}{!}{$
\begin{array}{c|cccc}
	\delta(1) & y_1 & y_1^2 & y_1^2y_2 & y_1^3 \\
	\hline
	1  &   &  &  & 1 \\
	x_2  &  & -1 & -1 & \\
	x_2^2  &  & -1 &  & \\
	x_2 + x_2^{2} + x_1  &  & 1 &  &
\end{array}
\hspace{0.2cm}
\begin{array}{c|cccc}
	\delta(x_1)  & y_1 & y_1^2 & y_1^2y_2 & y_1^3 \\
	\hline
	1  &  & 1 & 1 & \\
	x_2  &  &  &  & \\
	x_2^2  &  &  &  & -1 \\
	x_2 + x_2^{2} + x_1  &  &  &  & 1
\end{array}
\hspace{0.2cm}
\begin{array}{c|cccc}
	\delta(x_2) & y_1 & y_1^2 & y_1^2y_2 & y_1^3 \\
	\hline
	1  &  & -1 &  & \\
	x_2  &  &  &  & 1 \\
	x_2^2  &  &  & -1 & \\
	x_2 + x_2^{2} + x_1  & -1 &  &  &
\end{array}
$}

La première colonne de $B_2$ contient un seul coefficient non nul, en quatrième ligne, on peut donc supprimer les premières colonnes et les quatrièmes lignes. On obtient\\

$$
\begin{array}{c|ccc}
	\delta(1) & y_1^2 & y_1^2y_2 & y_1^3 \\
	\hline
	1  &  &  & 1 \\
	x_2  & -1 & -1 & \\
	x_2^2 & -1 &  &
\end{array}
\hspace{0.2cm}
\begin{array}{c|ccc}
	\delta(x_1) & y_1^2 & y_1^2y_2 & y_1^3 \\
	\hline
	1  & 1 & 1 & \\
	x_2  & 1 &  & -1\\
	x_2^2  &  &  & -1
\end{array}
\hspace{0.2cm}
\begin{array}{c|ccc}
	\delta(x_2) & y_1^2 & y_1^2y_2 & y_1^3 \\
	\hline
	1  & -1 &  & \\
	x_2  &  &  & 1 \\
	x_2^2  &  & -1 &
\end{array}
$$
A ce stade, la matrice de $\delta(1)$ est inversible, le processus de réduction est terminé. On vérifie facilement que les familles $\bold{x} = (1, x_2, x_2^{2})$ et $\bold{y} = (y_1, y_1^{2}, y_1^{3})$ sont des bases de $A$, dont les bases de Horner associées sont $\hat{\bold{x}} = (-x_2-x_2^{2}, -x_{2}, 1)$ et $\hat{\bold{y}} = (y_1^{3}, -y_1^{2}-y_1^{2}y_2, -y_1^{2})$. La dimension de $A$ est donc ici égale à $3$.\\

Plus généralement nous formulons la conjecture suivante (\cite{jpc} p.57)
\begin{conj}
Lorsque le processus de réduction est terminé, c'est-à-dire lorsque la matrice de $\delta(1)$ est inversible et que toutes les matrices $\delta(x_j), j=0, \cdots, n$ sont de même taille et indexées par des familles de polynômes $\bold{x, y}$, alors chacune des familles $\bold{x, y}$ est une base de $A$.
\end{conj}

\begin{rem}
Nous insistons particulièrement sur le fait que la conjecture précédente est valable uniquement dans l'hypothèse où l'idéal est zéro-dimensionnel. Lors de nos expériences nous avons pu observer que dans le cas contraire, il était possible d'obtenir en fin de processus des tailles de $\delta(1)$ différentes suivant que l'on considère les relations en $x$ ou en $y$ lors du processus de réduction. Dans ce cas, cette différence de taille finale est un phénomène qui reste à éclaircir.
\end{rem}


\subsubsection{Formule de Barnett et matrices compagnon}
Reprenons l'exemple \label{ex_bez_multi} et définissons $X_1, X_2$ par les quotients
\begin{equation}
	X_1 = B(x_1)B(1)^{-1} =
	\begin{bmatrix}
		0 & -1 & 0\\
		-1 & 0 & -1\\
		-1 & 0 & 0
	\end{bmatrix},\quad
	X_2 = B(x_2)B(1)^{-1} =
	\begin{bmatrix}
		0 & 0 & 1\\
		1 & 0 & 0\\
		0 & 1 & -1
	\end{bmatrix}
\end{equation}
On vérifie facilement que $X_1, X_2$ sont les matrices de multiplication par $x_1, x_2$ dans la base $\bold{x}$. $X_1, X_2$ sont donc des matrices compagnon, pour la base $\bold{x}$ et nous retrouvons donc dans la situation décrite à la section \ref{Bar}. Plus généralement nous avons
\begin{prop}
\label{Barnett_multi}
Lorsque le processus de réduction est terminé, que les matrices de Bezout s'écrivent dans des bases $\bold{x, y}$, alors les matrices compagnon $X_j$, c'est à dire les matrices de multiplication par $x_j$ dans la base $\bold{x}$, peuvent se calculer grâce aux {\bf formules de Barnett}
\begin{equation}
	X_j = B(x_j)B(1)^{-1}
\end{equation}
\end{prop}

\begin{rem}
D'une manière similaire au cas univariable, nous avons aussi, pour tout $j=1,\cdots,n$,\\
$B(x_j)^{T}B(1)^{-T}$ est la matrice de multiplication par $y_j$ dans la base $\bold{y}$ \\
$B(1)^{-1}B(x_j)$ est la matrice de multiplication par $x_j$ dans la base $\hat{\bold{x}}$ \\
$B(1)^{-T}B(x_j)^{T}$ est la matrice de multiplication par $y_j$ dans la base $\hat{\bold{y}}$
\end{rem}

\subsubsection{Calcul numérique des racines}
D'une manière similaire au cas univariable, comme indiqué dans la proposition \ref{compan2roots}, les racines numériques du système polynomial $f_1, \cdots, f_n$ s'obtiennent en calculant les valeurs propres des matrices compagnons (\cite{AS}).\\
Dans l'exemple précédent, les matrices $X_1, X_2$ fournissent les valeurs propres
$$
\begin{array}{c|c}
	x_1 & x_2 \\
	\hline
	-1.32472  & 0.75488 \\
	0.66236 + 0.56228i & -0.87744 + 0.74486i \\
	0.66236 - 0.56228i & -0.87744 - 0.74486i
\end{array}
$$
Comme l'algèbre $A$ est commutative, les matrices $X_1, X_2$ commutent et ont donc les mêmes vecteurs propres. Lors du calcul il faut donc faire attention d'ordonner les valeurs propres pour qu'elles correspondent aux mêmes vecteurs propres.
Dans l'exemple précédent on vérifie facilement que les couples $(x_1, x_2)$ ci-dessus sont bien numériquement solutions du système $f_1 = x_1^2 + x_1x_2^2 - 1, f_2 = x_1^2x_2 + x_1$.

\subsection{Structure bloc-triangulaire et rang numérique de $B(1)$}
Dans le processus de réduction, la première étape consiste à calculer le noyau de $B(1)$. Lorsque les coefficients des polynômes d'entrée sont entiers ou rationnels, ceci peut se faire de manière exacte au moyen d'un programme de calcul symbolique. La taille des entiers peut alors croître considérablement au cours des calculs et augmenter en conséquence le temps total de calcul et les besoins en mémoire du calculateur. Si par contre on veut effectuer l'ensemble des calculs en nombres flottants, ou si les coefficients d'entrée sont eux mêmes donnés sous forme numérique, alors on doit faire un calcul numérique du noyau.
La méthode éprouvée pour cela, implémentée dans des packages d'algèbre linéaire numérique comme Matlab/Octave, Numpy ou Julia, est d'effectuer une factorisation QR ``rank revealing'' de $B(1)$, que nous appellerons factorisation QRP, c'est-à-dire accompagnée de pivots sur les colonnnes. L'expérience montre que cette approche est souvent efficace mais peut s'avérer délicate à mettre en oeuvre si la taille de la matrice augmente. Montrons le sur un exemple.

\begin{exmp}
Nous choisissons $n = 4$,
\begin{align}
f_1 & = x_0^2x_1^2x_2x_3^2 - x_0^2x_1x_2^2x_3^2 - x_0^3x_1x_3^2 + x_0x_1^2x_2x_3^2 + x_0x_1^2x_3^2 + x_0^3x_2 - x_1^2x_3^2 + x_0^2x_2 + x_0x_1x_2 + x_1x_2x_3 + x_2^2 + x_2x_3 + 1,\nonumber\\
f_2 & = x_0^3x_1^2x_2x_3^2 + x_0^3x_1x_2^2 - x_0^2x_1^2x_2^2 + x_0^2x_1x_2x_3^2 - x_0^3x_1^2 + x_0^2x_1^2x_3 - x_0^3x_2x_3 + x_0x_1x_2^2x_3 + x_0^3x_3^2 - x_0x_1x_2x_3^2 + x_0x_1x_2^2 - x_0^3x_3 + x_0x_1x_2x_3 + x_1^2x_2x_3 - x_1^2x_3^2 - x_0^2x_1 + x_0x_1 + x_2^2 + x_3^2 - x_1 + 1,\nonumber\\
f_3 & =  -x_0^2x_1^2x_2^2x_3^2 - x_0^3x_1^2x_2x_3 - x_0^3x_2^2x_3^2 - x_0^3x_1x_2x_3 + x_0x_1x_2^2x_3^2 + x_0^3x_1 - x_1x_2x_3^2 + x_2^2x_3^2 + x_1^2x_3 - x_0x_2x_3 + x_0^2 + x_2^2 - x_0x_3 + x_1,\nonumber\\
f_4 & =  x_0^2x_1^2x_2^2x_3 + x_0^3x_1x_3^2 - x_0^3x_1x_2 - x_0^3x_3^2 - x_0x_1^2x_3^2 - x_0x_1x_2x_3 - x_1^2x_3^2 + x_1x_2^2 - x_1x_3^2 + x_3^2 - x_0 - 1 \nonumber
\end{align}
\end{exmp}
\begin{center}
\includegraphics[width=8cm]{B0.png}
\end{center}

Ce système est de multidegré $(3, 2, 2, 2)$, la matrice de Bezout $B(1)$ est de taille $576$. Si on effectue une factorisation QRP sur cette matrice, on sait que les éléments diagonaux seront triés en ordre décroissant, voir figure ci-dessous.
\begin{center}
\includegraphics[width=8cm]{qrp.png}
\end{center}
 On s'aperçoit alors que les derniers éléments non nuls décroissent très vite, et qu'il peut devenir difficile de choisir un seuil au dessus duquel les éléments diagonaux seront déclarés ``non nuls''.
En d'autres termes, le ``saut'' entre éléments ``non-nuls'' et éléments proches du epsilon de la machine a tendance à diminuer à mesure que la taille de la matrice augmente, ce qui rend le calcul du rang numérique difficile.\\
Nous pouvons cependant améliorer un peu la situation précédente, en exploitant une propriété de $B(1)$. En effet, en permutant lignes et colones de cette matrice d'une certaine façon, on peut arriver à une structure bloc-triangulaire de $B(1)$.
\begin{center}
\includegraphics[width=8cm]{B0_tri.png}
\end{center}
En factorisant la matrice bloc après bloc, les éléments diagonaux vont alors décroitre uniquement à l'intérieur de chaque bloc, ce qui permet à la fin de préserver un saut numérique plus grand que dans la première approche. Le graphique ci dessous montre la nouvelle disposition des éléments diagonaux pour le même exemple que précédemment, traité de la deuxième façon.
\begin{center}
\includegraphics[width=8cm]{bloc_triang.png}
\end{center}
On voit que, malgré une décroissance rapide des éléments diagonaux dans chaque bloc, la plus petite taille de ceux-ci permet au plus petit élément ``non-nul" d'être beaucoup plus grand que le epsilon machine, ce qui facilite le calcul du rang numérique.

\end{document}

\documentclass{standalone}
% Preamble
\begin{document}


\subsection{Formules de Barnett et structure de l'algèbre quotient.}
Dans l'hypothèse où l'idéal est zéro-dimensionel, la dimension de l'algèbre quotient $A = \C[\bold{x}]/\langle f\rangle$ est finie, ce qui assure l'existence d'une base et de matrices compagnon $X_1,\cdots, X_n$ (matrices des opérateurs de multiplication par les variables dans la base considérée). Nous allons montrer que le même procédé mis en oeuvre dans la section \ref{Bar_gen}, consistant en manipulations sur les colonnes des matrices de Bezout $B(1), B(x_1), \cdots, B(x_n)$, permet ici aussi de construire une base de $A$ ainsi que les matrices compagnons associées. Rappelons tout d'abord un certain nombre de propriétés algébriques du polynôme $\delta(1)$ et des matrices de Bezout $B(x_k)$.

\subsubsection{Propriétés algébriques du polynôme $\delta(1)$ et de la matrice $B(1)$}
Les propriétés qui suivent sont de nature algébrique et sont données sans démonstration. Le lecteur intéressé pourra consulter les détails dans \cite{jpc}. Comme dans la proposition \ref{Barnett}, définissons de nouvelles familles d'éléments de $A$ par les produits vecteur-matrice :
\begin{equation}
		\hat{\bold{x}}_k  =  \bold{x}B(x_k), \quad k=0\cdots n
\end{equation}
avec la convention de notation habituelle $\hat{\bold{x}}_0 = \hat{\bold{x}}$.
\begin{exmp}
En reprenant l'exemple \ref{bez_multi} nous avons
\begin{equation}
	\begin{array}{lll}
		\hat{\bold{x}}_0 & = & (0, -x_2 - x_1x_2^2, -1 - x_1x_2, x_1, -x_2, 1) \\
		\hat{\bold{x}}_1 & = & (0, 0, 0, -1 - x_2^2, -x_1x_2, x_1) \\
		\hat{\bold{x}}_2 & = & (-1 - x_1x_2, -x_2 - x_2^2 - x_1, - x_2 - x_1x_2^2, x_1x_2, -x_2^2, x_2)
	\end{array}
\end{equation}
\end{exmp}

\begin{prop}
\label{xj} (admise, démonstration dans \cite{jpc}).
Pour tout $k=1\cdots n$ on a
\begin{equation}
    \hat{\bold{x}}_0x_k = \hat{\bold{x}}_k
\end{equation}
\end{prop}
Les relations ci-dessus sont faciles à vérifier sur l'exemple \ref{bez_multi}. Jusqu'à maintenant les cas univariable et multivariable sont très similaires, sauf sur un point: dans le cas multivariable les familles $\bold{x}$ et $\hat{\bold{x}}$ ne sont plus nécessairement des bases de $A$. On a cependant la propriété suivante

\begin{prop} (admise, démonstration dans \cite{jpc}).
Chacune des familles $\bold{x}$ et $\hat{\bold{x}}$ est génératrice dans $A$.
\end{prop}

\subsubsection{Processus de réduction}
\label{sec:reduction_process}
La proposition précédente fournit un début de structure de l'algèbre $A$. Nous avons pour l'instant une famille génératrice $\bold{x}$ de $A$ ainsi que des matrices de Bezout $B(x_k), k = 0, \cdots, n$. Nous allons montrer comment, en appliquant le procédé matriciel décrit dans la section \ref{Bar_gen} à la famille génératrice $\bold{x}$ et aux matrices de Bezout $B(x_k)$, on peut fabriquer une base de $A$ et des matrices compagnon $X_k$. Illustrons les calculs à partir de l'exemple \ref{bez_multi}. Le rang de $B(1)$ est $5$. La première colonne de $B(x_1)$ est nulle mais celle de $B(x_2)$ ne l'est pas, ce qui fournit la relation dans le quotient $1 + x_1x_2 = 0$.
Multiplions $\bold{x}$ à droite par la matrice de Gauss $P$ dont la cinquième colonne vaut $(1, 0, 0, 0, 1, 0)^{T}$, et multiplions les matrices de Bezout à gauche par $P^{-1}$, ce qui revient à soustraire la cinquième ligne à la première.
Les matrices de Bezout $B(1), B(x_1), B(x_2)$ s'écrivent:
$$
\begin{array}{c|cccccc}
	B(1) & 1 & y_1 & y_1y_2 & y_1^2 & y_1^2y_2 & y_1^3 \\
	\hline
	1 &  &  &  &  &  & 1\\
	x_2 &  & -1 &  &  & -1 & \\
	x_2^2 &  &  &  &  &  & \\
	x_1 &  &  &  & 1 &  & \\
	1+x_1x_2 &  &  & -1 &  &  & \\
	x_1x_2^2 &  & -1 &  &  &  &
\end{array}$$
$$
\begin{array}{c|cccccc}
	B(x_1) & 1 & y_1 & y_1y_2 & y_1^2 & y_1^2y_2 & y_1^3 \\
	\hline
	1 &  &  &  & 1 & 1 & \\
	x_2 &  &  &  &  &  & \\
	x_2^2 &  &  &  &  &  & \\
	x_1 &  &  &  &  &  & 1\\
	1+x_1x_2 &  &  &  &  & -1 & \\
	x_1x_2^2 &  &  &  & -1 &  &
\end{array}
\hspace{0.2cm}
\begin{array}{c|cccccc}
	B(x_2) & 1 & y_1 & y_1y_2 & y_1^2 & y_1^2y_2 & y_1^3 \\
	\hline
	1 &  &  &  & -1 &  & \\
	x_2 &  & -1 & -1 &  &  & 1\\
	x_2^2 &  & -1 &  &  & -1 & \\
	x_1 &  & -1 &  &  &  & \\
	1+x_1x_2 & -1 &  &  & 1 &  & \\
	x_1x_2^2 &  &  & -1 &  &  &
\end{array}
$$
La première colonne de $B(x_2)$ contient maintenant un seul coefficient non nul, indexé par $1 + x_1x_2$. On peut donc, en projetant les trois bezoutiens sur $A_x$, supprimer la première colonne et la cinquième ligne dans les trois matrices:
$$
\begin{array}{c|ccccc}
	B(1) & y_1 & y_1y_2 & y_1^2 & y_1^2y_2 & y_1^3 \\
	\hline
	1  &  &  &  &  & 1 \\
	x_2  & -1 &  &  & -1 & \\
	x_2^2  &  &  &  &  & \\
	x_1  &  &  & 1 &  & \\
	x_1x_2^2  & -1 &  &  &  &
\end{array}$$
$$
\begin{array}{c|ccccc}
	B(x_1)  & y_1 & y_1y_2 & y_1^2 & y_1^2y_2 & y_1^3 \\
	\hline
	1  &  &  & 1 & 1 & \\
	x_2  &  &  &  &  & \\
	x_2^2  &  &  &  &  & \\
	x_1  &  &  &  &  & 1 \\
	x_1x_2^2  &  &  & -1 &  &
\end{array}
\hspace{0.2cm}
\begin{array}{c|ccccc}
	B(x_2) & y_1 & y_1y_2 & y_1^2 & y_1^2y_2 & y_1^3 \\
	\hline
	1  &  &  & -1 &  & \\
	x_2  & -1 & -1 &  &  & 1 \\
	x_2^2  & -1 &  &  & -1 & \\
	x_1  & -1 &  &  &  & \\
	x_1x_2^2 &  & -1 &  &  &
\end{array}
$$

La deuxième colonne de $B(1)$ est nulle, celle de $B(x_2)$ ne l'est pas. La relation est $x_2 + x_1x_2^{2} = 0$. La matrice $P$ est définie par sa cinquième colonne $(0, 1, 0, 0, 1)^{T}$. Le vecteur $\bold{x}$ devient $(1, x_2, x_2^{2}, x_1, x_2 + x_1x_2^{2})$. On soustrait la cinquième ligne à la deuxième. Les bezoutiens s'écrivent:
$$
\begin{array}{c|ccccc}
	B(1) & y_1 & y_1y_2 & y_1^2 & y_1^2y_2 & y_1^3 \\
	\hline
	1  &  &  &  &  & 1 \\
	x_2  &  &  &  & -1 & \\
	x_2^2  &  &  &  &  & \\
	x_1  &  &  & 1 &  & \\
	x_2 + x_1x_2^2  & -1 &  &  &  &
\end{array}$$
$$
\begin{array}{c|ccccc}
	B(x_1)  & y_1 & y_1y_2 & y_1^2 & y_1^2y_2 & y_1^3 \\
	\hline
	1  &  &  & 1 & 1 & \\
	x_2  &  &  & 1 &  & \\
	x_2^2  &  &  &  &  & \\
	x_1  &  &  &  &  & 1 \\
	x_2 + x_1x_2^2  &  &  & -1 &  &
\end{array}
\hspace{0.2cm}
\begin{array}{c|ccccc}
	B(x_2) & y_1 & y_1y_2 & y_1^2 & y_1^2y_2 & y_1^3 \\
	\hline
	1  &  &  & -1 &  & \\
	x_2  & -1 &  &  &  & 1 \\
	x_2^2  & -1 &  &  & -1 & \\
	x_1  & -1 &  &  &  & \\
	x_2 + x_1x_2^2 &  & -1 &  &  &
\end{array}
$$

La deuxième colonne de $B_2$ contient un seul coefficient non nul, en cinquième ligne, on peut donc supprimer les deuxièmes colonnes et les cinquièmes lignes:

$$
\begin{array}{c|cccc}
	B(1) & y_1 & y_1^2 & y_1^2y_2 & y_1^3 \\
	\hline
	1  &   &  &  & 1 \\
	x_2  &  &  & -1 & \\
	x_2^2  &  &  &  & \\
	x_1  &  & 1 &  &
\end{array}
\hspace{0.2cm}
\begin{array}{c|cccc}
	B(x_1)  & y_1 & y_1^2 & y_1^2y_2 & y_1^3 \\
	\hline
	1  &  & 1 & 1 & \\
	x_2  &  & 1 &  & \\
	x_2^2  &  &  &  & \\
	x_1  &  &  &  & 1
\end{array}
\hspace{0.2cm}
\begin{array}{c|cccc}
	B(x_2) & y_1 & y_1^2 & y_1^2y_2 & y_1^3 \\
	\hline
	1  &  & -1 &  & \\
	x_2  & -1 &  &  & 1 \\
	x_2^2  & -1 &  & -1 & \\
	x_1  & -1 &  &  &
\end{array}
$$
La première colonne de $B(1)$ est nulle, celle de $B(x_2)$ ne l'est pas. La relation est $x_2 + x_2^{2} + x_1 = 0$. La matrice $P$ est définie par sa quatrième colonne $(0, 1, 1, 1)^{T}$. Le vecteur $\bold{x}$ devient $(1, x_2, x_2^{2},  x_2 + x_2^{2} + x_1)$. On soustrait la quatrième ligne à la deuxième et à la troisième. Les bezoutiens s'écrivent:

$$
\begin{array}{c|cccc}
	B(1) & y_1 & y_1^2 & y_1^2y_2 & y_1^3 \\
	\hline
	1  &   &  &  & 1 \\
	x_2  &  & -1 & -1 & \\
	x_2^2  &  & -1 &  & \\
	x_2 + x_2^{2} + x_1  &  & 1 &  &
\end{array}$$
$$
\begin{array}{c|cccc}
	B(x_1)  & y_1 & y_1^2 & y_1^2y_2 & y_1^3 \\
	\hline
	1  &  & 1 & 1 & \\
	x_2  &  &  &  & \\
	x_2^2  &  &  &  & -1 \\
	x_2 + x_2^{2} + x_1  &  &  &  & 1
\end{array}
\hspace{0.2cm}
\begin{array}{c|cccc}
	B(x_2) & y_1 & y_1^2 & y_1^2y_2 & y_1^3 \\
	\hline
	1  &  & -1 &  & \\
	x_2  &  &  &  & 1 \\
	x_2^2  &  &  & -1 & \\
	x_2 + x_2^{2} + x_1  & -1 &  &  &
\end{array}$$

La première colonne de $B_2$ contient un seul coefficient non nul, en quatrième ligne, on peut donc supprimer les premières colonnes et les quatrièmes lignes:
$$
\begin{array}{c|ccc}
	B(1) & y_1^2 & y_1^2y_2 & y_1^3 \\
	\hline
	1  &  &  & 1 \\
	x_2  & -1 & -1 & \\
	x_2^2 & -1 &  &
\end{array}
\hspace{0.2cm}
\begin{array}{c|ccc}
	B(x_1) & y_1^2 & y_1^2y_2 & y_1^3 \\
	\hline
	1  & 1 & 1 & \\
	x_2  & 1 &  & -1\\
	x_2^2  &  &  & -1
\end{array}
\hspace{0.2cm}
\begin{array}{c|ccc}
	B(x_2) & y_1^2 & y_1^2y_2 & y_1^3 \\
	\hline
	1  & -1 &  & \\
	x_2  &  &  & 1 \\
	x_2^2  &  & -1 &
\end{array}$$
A ce stade, la matrice $B(1)$ est inversible et le processus de réduction est donc terminé. On vérifie que les familles $\bold{x} = (1, x_2, x_2^{2})$ et $\bold{y} = (y_1, y_1^{2}, y_1^{3})$ sont des bases de $A$, dont les bases de Horner associées sont $\hat{\bold{x}} = (-x_2-x_2^{2}, -x_{2}, 1)$ et
$\hat{\bold{y}} = (y_1^{3}, -y_1^{2}-y_1^{2}y_2, -y_1^{2})$. La dimension de $A$ est ici égale à~$3$.
D'une façon générale nous avons (\cite{jpc} p.57, \cite{bm}, \cite{tm})
\begin{prop}
	\label{conjecture}
Lorsque le processus de réduction est terminé, c'est-à-dire lorsque la matrice $B(1)$ est inversible et que toutes les matrices $B(x_k), k=0, \cdots, n$ sont de même taille et indexées par des familles de polynômes $\bold{x, y}$, alors chacune des familles $\bold{x, y}$ est une base de $A$.
\end{prop}

\begin{rem}
Nous insistons sur le fait que la proposition précédente est valable uniquement dans l'hypothèse où l'idéal est zéro-dimensionnel. Lors de nos expériences nous avons pu observer que dans le cas contraire, il est possible d'obtenir à la fin du processus de réduction des matrices $B(1)$ de tailles différentes suivant que l'on utilise les relations en $x$ ou en $y$ lors du processus de réduction. Le cas échéant, cette différence de taille finale est un phénomène qui reste à éclaircir.
\end{rem}

\subsubsection{Formules de Barnett et matrices compagnon}
Reprenons l'exemple précédent et définissons les matrices $X_1, X_2$ par les quotients
\begin{equation}
	X_1 = B(x_1)B(1)^{-1} =
	\begin{bmatrix}
		0 & -1 & 0\\
		-1 & 0 & -1\\
		-1 & 0 & 0
	\end{bmatrix},\quad
	X_2 = B(x_2)B(1)^{-1} =
	\begin{bmatrix}
		0 & 0 & 1\\
		1 & 0 & 0\\
		0 & 1 & -1
	\end{bmatrix}
\end{equation}
On vérifie que $X_1, X_2$ sont les matrices de multiplication par les variables $x_1, x_2$ dans la base $\bold{x}$ et sont donc les matrices compagnon associées à la base $\bold{x}$. D'une façon générale nous avons:
\begin{prop}
\label{Barnett_multi}
Lorsque le processus de réduction est terminé et que les matrices de Bezout sont écrites dans des bases $\bold{x, y}$, alors les matrices compagnon $X_j$, c'est à dire les matrices de multiplication par $x_j$ dans la base $\bold{x}$, peuvent se calculer grâce aux {\bf formules de Barnett}
\begin{equation}
	X_j = B(x_j)B(1)^{-1}
\end{equation}
\end{prop}

\begin{rem}
Comme dans le cas univariable nous avons pour tout $j=1,\cdots,n$,\\
$B(x_j)^{T}B(1)^{-T}$ est la matrice de multiplication par $y_j$ dans la base $\bold{y}$ \\
$B(1)^{-1}B(x_j)$ est la matrice de multiplication par $x_j$ dans la base $\hat{\bold{x}}$ \\
$B(1)^{-T}B(x_j)^{T}$ est la matrice de multiplication par $y_j$ dans la base $\hat{\bold{y}}$
\end{rem}

\subsubsection{Calcul numérique des racines}
Comme dans le cas univariable (voir Proposition \ref{compan2roots}) les racines du système polynomial $f_1, \cdots, f_n$ s'obtiennent numériquement en calculant les valeurs propres des matrices compagnons (\cite{AS}). Dans cet exemple les matrices $X_1, X_2$ fournissent les valeurs propres
$$
\begin{array}{c|c}
	x_1 & x_2 \\
	\hline
	-1.32472  & 0.75488 \\
	0.66236 + 0.56228i & -0.87744 + 0.74486i \\
	0.66236 - 0.56228i & -0.87744 - 0.74486i
\end{array}
$$
Puisque l'algèbre $A$ est commutative, les matrices $X_1, X_2$ commutent et ont donc les mêmes vecteurs propres. Lors du calcul il faut donc faire attention d'ordonner les valeurs propres pour qu'elles correspondent aux mêmes vecteurs propres. Dans l'exemple précédent on vérifie facilement que les couples $(x_1, x_2)$ ci-dessus sont bien des approximations des racines du système $f_1 = x_1^2 + x_1x_2^2 - 1, f_2 = x_1^2x_2 + x_1$.


\end{document}

\documentclass{standalone}
% Preamble
\begin{document}

\subsection{Structure bloc-triangulaire et rang numérique de $B(1)$}
Dans le processus de réduction, la première étape consiste à calculer le noyau de $B(1)$. Lorsque les coefficients des polynômes d'entrée sont entiers ou rationnels, ceci peut se faire de manière exacte au moyen d'un programme de calcul symbolique. La taille des entiers peut alors croître considérablement au cours des calculs et augmenter en conséquence le temps total de calcul et les besoins en mémoire du calculateur. Si par contre on veut effectuer l'ensemble des calculs en nombres flottants, ou si les coefficients d'entrée sont eux mêmes donnés sous forme numérique, alors on doit faire un calcul numérique du noyau. La méthode éprouvée pour cela, implémentée dans des packages d'algèbre linéaire numérique comme Matlab/Octave, Numpy ou Julia, est d'effectuer une factorisation QR ``rank revealing'' de $B(1)$, que nous appellerons factorisation QRP, c'est-à-dire accompagnée de pivots sur les colonnnes. L'expérience montre que cette approche est souvent efficace mais peut s'avérer délicate à mettre en oeuvre si la taille de la matrice augmente. Montrons le sur un exemple. Nous choisissons $n = 4$ et un système polynomial $f$ de multidegré $d = (2, 2, 2, 2)$. Chaque polynôme possède $m = 15$ monômes. Les degrés sont choisis aléatoirement suivant une loi uniforme entre $0$ et $2$. Les coefficients, entiers, sont choisis aléatoirement suivant une loi uniforme entre $-5$ et $5$. La matrice de Bezout $B(1)$ est de taille $384$ et on peut voir apparaitre une certaine structure, comme le montre la figure ci-dessous.
% \begin{wrapfigure}{r}{0.5\textwidth}
  \begin{center}
    \includegraphics[width=0.48\textwidth]{../png/bez.png}
  \end{center}
% \end{wrapfigure}
Cependant il parait difficile d'exploiter cette structure pour le calcul du rang de la matrice $B(1)$. On doit donc avoir recours à une méthode numérique générale, par exemple une factorisation SVD ou une factorisation QRP ``rank revealing''. Choisissons cette deuxième méthode. Les termes diagonaux du facteur triangulaire $R$ seront triés en ordre décroissant, comme le montre la figure ci-dessous.
% \begin{wrapfigure}{l}{0.5\textwidth}
  \begin{center}
    \includegraphics[width=0.48\textwidth]{../png/diagR.png}
  \end{center}
% \end{wrapfigure}
 On s'aperçoit que les derniers termes non nuls décroissent très vite, et qu'il peut devenir difficile de choisir un seuil au dessus duquel les termes diagonaux seront déclarés ``non nuls''. Le ``saut'' entre termes ``non-nuls'' et termes proches du epsilon machine a tendance à diminuer à mesure que la taille de la matrice augmente, ce qui rend le calcul du rang numérique difficile. Pour cet exemple, voici la table des termes de la diagonale entre les indices $294$ et $297$.
 $$
 \begin{array}{c|c}
 	i & \vert R_{i,i}\vert \\
 	\hline
 	294  & 1e-4 \\
 	295 & 1e-6 \\
 	296 & 1e-11 \\
 	297 & 1e-11 \\
 \end{array}
 $$
On constate qu'il est difficile, à partir de ces valeurs de fixer un seuil en dessous duquel les termes peuvent être considérés comme négligeables. On aurait tendance à fixer naturellement le seuil entre les indices $295$ et $296$ ce qui donne $295$ comme rang numérique de $B(1)$. C'est la valeur fournie par la fonction rank de numpy. Or $B(1)$ est une matrice à coefficients entiers et on peut donc calculer son rang exact au moyen d'un logiciel de calcul symbolique. Pour le logiciel Sage le rang de la matrice à coefficient entiers $B(1)$ vaut $296$, en contradiction avec la valeur obtenue par la factorisation numérique QRP.

Nous pouvons cependant améliorer, dans une certaine mesure, la situation précédente en exploitant une propriété de $B(1)$. En effet, en permutant lignes et colones de cette matrice d'une certaine façon, on peut arriver à une structure bloc-triangulaire de $B(1)$.
\begin{center}
\includegraphics[width=8cm]{../png/beztri.png}
\end{center}
En appliquant à la matrice une factorisation QRP bloc après bloc, les éléments diagonaux vont alors décroitre uniquement à l'intérieur de chaque bloc, ce qui permet de préserver un saut numérique plus grand que dans la première approche. Le graphique ci dessous montre la nouvelle disposition des éléments diagonaux pour le même exemple que précédemment, traité de la deuxième façon.
\begin{center}
\includegraphics[width=8cm]{../png/diagRtri.png}
\end{center}
On voit que, malgré une décroissance rapide des éléments diagonaux dans chaque bloc, la plus petite taille de ceux-ci permet au plus petit élément ``non-nul" d'être beaucoup plus grand que le epsilon machine, ce qui facilite le calcul du rang numérique.

\end{document}

\documentclass{standalone}
% Preamble
\begin{document}

  \subsection{Calcul numérique des racines}
  Nous reprenons l'exemple ci-dessus. La matrice de Bezout $B(1)$, à coefficients entiers, est de taille \input{../txt/Dx.txt}. Après réductions on trouve que la dimension du quotient $A$ est \input{../txt/dim.txt}. En calculant numériquement les valeurs propres des matrices compagnon $X_j = B(x_j)B(1)^{-1}$ on obtient les racines du système polynômial $f$. On vérifie la qualité de chacune des racines obtenues en lui appliquant les polynômes $f_i, i=1,\cdots,n$. Les résultats sont représentés sous forme d'histogramme o\`u le logarithme décimal de l'erreur est porté en abscisse.
\begin{figure}[!ht]
  \includegraphics[height=8cm, width=1.2\textwidth]{../png/roots.png}
  \caption{histograms}
\end{figure}
Sur la figure de gauche nous avons les erreurs correspondants à l'algorithme o\`u le processus de réduction est effectué en arithmétique exacte (logiciel Sage, temps de calcul
\input{../txt/sage_reduct_time.txt} secondes). Sur la figure de droite le processus de réduction est effectué en arithmétique flottante (logiciel Octave, temps de calcul
\input{../txt/octave_reduct_time.txt} secondes). On constate que le temps de calcul en arithmétique flottante est plus court mais au prix d'une dégradation sensible de la qualité des résultats, bien visible sur le plot des histogrammes.
\begin{center}
\begin{tabular}{p{8cm}lr}
  Processus & Software & Timing \\ \hline
  Calcul matrices de Bezout & NumPy & 0.0 secs \\ \hline
  Réduction matrices de Bezout & SageMath & $\input{../txt/sage_reduct_time.txt}$ ms \\
                              & Octave & $\input{../txt/octave_reduct_time.txt}$ ms \\ \hline
  Calcul valeurs propres & SciPy & 0.0 secs \\ \hline
\end{tabular}
\end{center}


\end{document}

\documentclass{standalone}
% Preamble
\begin{document}
\section{Conclusion et perspectives}
Nous avons proposé une méthode de résolution numérique des systèmes polynômiaux en intersection complète. Cette méthode utilise exclusivement des techniques d'algèbre linéaire numérique. Le principe de la méthode repose sur une conjecture de nature algébrique mais il est facile de tester si les racines obtenues sont numériquement correctes et les racines non satisfaisantes peuvent être écartées.


\end{document}

\documentclass{standalone}
% Preamble
\begin{document}

  \begin{thebibliography}{00}

  \bibitem{AS}
  {Auzinger, Stetter}, {A note }, {SIAM }, {19?}

  \bibitem{Barnett}
  {S. Barnett}, {A note on the Bezoutian matrix}, {SIAM J. Appl. Math., 22:84-86}, {1972}

  \bibitem{Golub}
  {G.H. Golub, C.F. van Loan}, {Matrix Computations}, {The Johns Hopkins University Press}, {1989}.

  \bibitem{jpc}
  {J.P. Cardinal}, {Dualité et algorithmes itératifs pour la résolution de systèmes polynomiaux}, {Thèse présentée devant l'université de Rennes I}, {1993}.

  \bibitem{CM}
  {J.P. Cardinal, B. Mourrain}, {Algebraic approach of residues and applications}, {in: J. Renegar, M. Shub, S. Smale (Eds.), Proc. AMS-SIAM Summer Seminar on Math. of Numerical Analysis, Park City, Utah, 1995, Lectures in Appl. Math., 32, Amer. Math. Soc, Providence, pp. 189-210}, {1996}

  \bibitem{clo}
  {D. Cox, J. Little, D. O'Shea}{Ideals, Varieties and Algorithms}
  \end{thebibliography}

\end{document}


\end{document}
